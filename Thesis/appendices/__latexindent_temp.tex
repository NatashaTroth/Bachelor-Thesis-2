
\subsubsection{Perplexity = 5}
%------------------ PERPLEXITY 10: ------------------
% -- 1h, perp 5 --
\begin{figure}[H]
  \centering
  \begin{subfigure}{.5\textwidth}
    \centering
    \includegraphics[width=0.9\textwidth]{./images/tsneParametersTest/perplexity/perp5-1hTSNE.png}
  \end{subfigure}%
  \begin{subfigure}{.5\textwidth}
    \centering
    \includegraphics[width=0.9\textwidth]{./images/tsneParametersTest/perplexity/perp5-1hDBSCAN.png}
  \end{subfigure}
	\caption{\textbf{1h} data files, t-SNE calculated with the following parameters: \textbf{perplexity=5}, n\_iter=5000, learning\_rate=50}
	\label{figure:1hperp5TSNE}
\end{figure}


% -- 3h, perp 5 --
\begin{figure}[H]
	\centering
	
  \centering
	\begin{subfigure}{.5\textwidth}
    \centering
    \includegraphics[width=0.9\textwidth]{./images/tsneParametersTest/perplexity/perp5-3hTSNE.png}
  \end{subfigure}%
  \begin{subfigure}{.5\textwidth}
    \centering
    \includegraphics[width=0.9\textwidth]{./images/tsneParametersTest/perplexity/perp5-3hDBSCAN.png}
	\end{subfigure}
	\caption{\textbf{3h} data files, t-SNE calculated with the following parameters: \textbf{perplexity=5}, n\_iter=5000, learning\_rate=50}
  \label{figure:3hperp5TSNE}
\end{figure}


%------------------ PERPLEXITY 10: ------------------
\subsubsection{Perplexity = 10}
% -- 1h, perp 10 --
\begin{figure}[H]
  \centering
  \begin{subfigure}{.5\textwidth}
    \centering
    \includegraphics[width=0.9\textwidth]{./images/tsneParametersTest/perplexity/perp10-1hTSNE.png}
  \end{subfigure}%
  \begin{subfigure}{.5\textwidth}
    \centering
    \includegraphics[width=0.9\textwidth]{./images/tsneParametersTest/perplexity/perp10-1hDBSCAN.png}
  \end{subfigure}
	\caption{\textbf{1h} data files, t-SNE calculated with the following parameters: \textbf{perplexity=10}, n\_iter=5000, learning\_rate=50}
  \label{figure:1hperp10TSNE}
\end{figure}

% -- 3h, perp 10 --
\begin{figure}[H]
  \centering
	\begin{subfigure}{.5\textwidth}
    \centering
    \includegraphics[width=0.9\textwidth]{./images/tsneParametersTest/perplexity/perp10-3hTSNE.png}
  \end{subfigure}%
  \begin{subfigure}{.5\textwidth}
    \centering
    \includegraphics[width=0.9\textwidth]{./images/tsneParametersTest/perplexity/perp10-3hDBSCAN.png}
	\end{subfigure}
	\caption{\textbf{3h} data files, t-SNE calculated with the following parameters: \textbf{perplexity=10}, n\_iter=5000, learning\_rate=50}
  \label{figure:3hperp10TSNE}
\end{figure}

%------------------ PERPLEXITY 20: ------------------
\subsubsection{Perplexity = 20}
% -- 1h, perp 20 --
\begin{figure}[H]
  \centering
  \begin{subfigure}{.5\textwidth}
    \centering
    \includegraphics[width=0.9\textwidth]{./images/tsneParametersTest/perplexity/perp20-1hTSNE.png}
  \end{subfigure}%
  \begin{subfigure}{.5\textwidth}
    \centering
    \includegraphics[width=0.9\textwidth]{./images/tsneParametersTest/perplexity/perp20-1hDBSCAN.png}
  \end{subfigure}
	\caption{\textbf{1h} data files, t-SNE calculated with the following parameters: \textbf{perplexity=20}, n\_iter=5000, learning\_rate=50}
  \label{figure:1hperp20TSNE}
\end{figure}

% -- 3h, perp 20 --
\begin{figure}[H]
  \centering
	\begin{subfigure}{.5\textwidth}
    \centering
    \includegraphics[width=0.9\textwidth]{./images/tsneParametersTest/perplexity/perp20-3hTSNE.png}
  \end{subfigure}%
  \begin{subfigure}{.5\textwidth}
    \centering
    \includegraphics[width=0.9\textwidth]{./images/tsneParametersTest/perplexity/perp20-3hDBSCAN.png}
	\end{subfigure}
	\caption{\textbf{3h} data files, t-SNE calculated with the following parameters: \textbf{perplexity=20}, n\_iter=5000, learning\_rate=50}
  \label{figure:3hperp20TSNE}
\end{figure}



%------------------ PERPLEXITY 30: ------------------
\subsubsection{Perplexity = 30}
% -- 1h, perp 30 --
\begin{figure}[H]
  \centering
  \begin{subfigure}{.5\textwidth}
    \centering
    \includegraphics[width=0.9\textwidth]{./images/tsneParametersTest/perplexity/perp30-1hTSNE.png}
  \end{subfigure}%
  \begin{subfigure}{.5\textwidth}
    \centering
    \includegraphics[width=0.9\textwidth]{./images/tsneParametersTest/perplexity/perp30-1hDBSCAN.png}
  \end{subfigure}
	\caption{\textbf{1h} data files, t-SNE calculated with the following parameters: \textbf{perplexity=30}, n\_iter=5000, learning\_rate=50}
  \label{figure:1hperp30TSNE}
\end{figure}

% -- 3h, perp 30 --
\begin{figure}[H]
  \centering
	\begin{subfigure}{.5\textwidth}
    \centering
    \includegraphics[width=0.9\textwidth]{./images/tsneParametersTest/perplexity/perp30-3hTSNE.png}
  \end{subfigure}%
  \begin{subfigure}{.5\textwidth}
    \centering
    \includegraphics[width=0.9\textwidth]{./images/tsneParametersTest/perplexity/perp30-3hDBSCAN.png}
	\end{subfigure}
	\caption{\textbf{3h} data files, t-SNE calculated with the following parameters: \textbf{perplexity=30}, n\_iter=5000, learning\_rate=50}
  \label{figure:3hperp30TSNE}
\end{figure}



%------------------ PERPLEXITY 40: ------------------
\subsubsection{Perplexity = 40}
% -- 1h, perp 40 --
\begin{figure}[H]
  \centering
  \begin{subfigure}{.5\textwidth}
    \centering
    \includegraphics[width=0.9\textwidth]{./images/tsneParametersTest/perplexity/perp40-1hTSNE.png}
  \end{subfigure}%
  \begin{subfigure}{.5\textwidth}
    \centering
    \includegraphics[width=0.9\textwidth]{./images/tsneParametersTest/perplexity/perp40-1hDBSCAN.png}
  \end{subfigure}
	\caption{\textbf{1h} data files, t-SNE calculated with the following parameters: \textbf{perplexity=40}, n\_iter=5000, learning\_rate=50}
  \label{figure:1hperp40TSNE}
\end{figure}

% -- 3h, perp 40 --
\begin{figure}[H]
  \centering
	\begin{subfigure}{.5\textwidth}
    \centering
    \includegraphics[width=0.9\textwidth]{./images/tsneParametersTest/perplexity/perp40-3hTSNE.png}
  \end{subfigure}%
  \begin{subfigure}{.5\textwidth}
    \centering
    \includegraphics[width=0.9\textwidth]{./images/tsneParametersTest/perplexity/perp40-3hDBSCAN.png}
	\end{subfigure}
	\caption{\textbf{3h} data files, t-SNE calculated with the following parameters: \textbf{perplexity=40}, n\_iter=5000, learning\_rate=50}
  \label{figure:3hperp40TSNE}
\end{figure}



%------------------ PERPLEXITY 45: ------------------
\subsubsection{Perplexity = 45}
% -- 1h, perp 45 --
\begin{figure}[H]
  \centering
  \begin{subfigure}{.5\textwidth}
    \centering
    \includegraphics[width=0.9\textwidth]{./images/tsneParametersTest/perplexity/perp45-1hTSNE.png}
  \end{subfigure}%
  \begin{subfigure}{.5\textwidth}
    \centering
    \includegraphics[width=0.9\textwidth]{./images/tsneParametersTest/perplexity/perp45-1hDBSCAN.png}
  \end{subfigure}
	\caption{\textbf{1h} data files, t-SNE calculated with the following parameters: \textbf{perplexity=45}, n\_iter=5000, learning\_rate=50}
  \label{figure:1hperp45TSNE}
\end{figure}

% -- 3h, perp 45 --
\begin{figure}[H]
  \centering
	\begin{subfigure}{.5\textwidth}
    \centering
    \includegraphics[width=0.9\textwidth]{./images/tsneParametersTest/perplexity/perp45-3hTSNE.png}
  \end{subfigure}%
  \begin{subfigure}{.5\textwidth}
    \centering
    \includegraphics[width=0.9\textwidth]{./images/tsneParametersTest/perplexity/perp45-3hDBSCAN.png}
	\end{subfigure}
	\caption{\textbf{3h} data files, t-SNE calculated with the following parameters: \textbf{perplexity=45}, n\_iter=5000, learning\_rate=50}
  \label{figure:3hperp45TSNE}
\end{figure}



%------------------ PERPLEXITY 50: ------------------
\subsubsection{Perplexity = 50}
% -- 1h, perp 50 --
\begin{figure}[H]
  \centering
  \begin{subfigure}{.5\textwidth}
    \centering
    \includegraphics[width=0.9\textwidth]{./images/tsneParametersTest/perplexity/perp50-1hTSNE.png}
  \end{subfigure}%
  \begin{subfigure}{.5\textwidth}
    \centering
    \includegraphics[width=0.9\textwidth]{./images/tsneParametersTest/perplexity/perp50-1hDBSCAN.png}
  \end{subfigure}
	\caption{\textbf{1h} data files, t-SNE calculated with the following parameters: \textbf{perplexity=50}, n\_iter=5000, learning\_rate=50}
  \label{figure:1hperp50TSNE}
\end{figure}

% -- 3h, perp 50 --
\begin{figure}[H]
  \centering
	\begin{subfigure}{.5\textwidth}
    \centering
    \includegraphics[width=0.9\textwidth]{./images/tsneParametersTest/perplexity/perp50-3hTSNE.png}
  \end{subfigure}%
  \begin{subfigure}{.5\textwidth}
    \centering
    \includegraphics[width=0.9\textwidth]{./images/tsneParametersTest/perplexity/perp50-3hDBSCAN.png}
	\end{subfigure}
	\caption{\textbf{3h} data files, t-SNE calculated with the following parameters: \textbf{perplexity=50}, n\_iter=5000, learning\_rate=50}
  \label{figure:3hperp50TSNE}
\end{figure}





\subsubsection{Perplexity Comparison Results (Average of two different t-SNE runs)}
\label{appendix:compareAveragePerplexity}

\begin{figure}[H]
  \centering
  \includegraphics[width=0.8\textwidth]{./images/tsneParametersTest/perplexity/perplexityEvaluationScoresAverage.png}
  \caption{Comparison of the average of two Silhouette Coefficients, Davies-Bouldin Indices, and Caliński-Harabasz Indices for different t-SNE \textbf{perplexities} in steps of 5 and 10. The lighter green highlighted values indicate the best values of that file aggregation (1h or 3h files). The dark green highlighted values illustrate the overall best values over all files (1h and 3h files).}
  \label{figure:perplexityEvaluationScoresAverage}
\end{figure}

\begin{figure}[H]
  \centering
  \includegraphics[width=0.4\textwidth]{./images/tsneParametersTest/perplexity/perplexityEvaluationScoresDetailedAverage.png}
  \caption{Comparison of the average of two evaluation scores of the top three perplexity candidates 40, 45, and 50. The lighter green highlighted values indicate the best values of that file aggregation (1h or 3h files). The dark green highlighted values illustrate the overall best values over all files (1h and 3h files).}
  \label{figure:perplexityEvaluationScoresDetailedAverage}
\end{figure}


\clearpage