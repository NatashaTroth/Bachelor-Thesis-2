
% In the following figures, the t-SNE results of different perplexitites are compared, for the different time length files (1h and 3h), using the first columns of each feature (1h: first 15 minutes, 3h: fist 30 minutes ). The left scatter plots depict t-SNE results, the right scatter plots visualise DBSCAN clusterings of t-SNE results).

%------------------ LEARNING RATE 10: ------------------
\subsubsection{Learning Rate = 10}
% -- 1h, lr 10 --
\begin{figure}[H]
  \centering
  \begin{subfigure}{.5\textwidth}
    \centering
    \includegraphics[width=0.9\textwidth]{./images/tsneParametersTest/learningRate/lr10-1hTSNE.png}
  % \caption{}
  % \label{figure:}
  \end{subfigure}%
  \begin{subfigure}{.5\textwidth}
    \centering
    \includegraphics[width=0.9\textwidth]{./images/tsneParametersTest/learningRate/lr10-1hDBSCAN.png}
    % \caption{}
    % \label{figure:}
  \end{subfigure}
	\caption{\textbf{1h} data files, t-SNE calculated with the following parameters: perplexity=40, n\_iter=5000, \textbf{learning\_rate=10}}
	\label{figure:1hlr10TSNE}
\end{figure}

% -- 3h, lr 10 --
\begin{figure}[H]
	\centering
	
  \centering
	\begin{subfigure}{.5\textwidth}
    \centering
    \includegraphics[width=0.9\textwidth]{./images/tsneParametersTest/learningRate/lr10-3hTSNE.png}
  % \caption{}
  % \label{figure:}
  \end{subfigure}%
  \begin{subfigure}{.5\textwidth}
    \centering
    \includegraphics[width=0.9\textwidth]{./images/tsneParametersTest/learningRate/lr10-3hDBSCAN.png}
    % \caption{}
    % \label{figure:}
	\end{subfigure}
	\caption{\textbf{3h} data files, t-SNE calculated with the following parameters: perplexity=40, n\_iter=5000, \textbf{learning\_rate=10}}
  \label{figure:3hlr10TSNE}
\end{figure}

%------------------ LEARNING RATE 200: ------------------
\subsubsection{Learning Rate = 200}
% -- 1h, lr 200 --
\begin{figure}[H]
  \centering
  \begin{subfigure}{.5\textwidth}
    \centering
    \includegraphics[width=0.9\textwidth]{./images/tsneParametersTest/learningRate/lr200-1hTSNE.png}
  % \caption{}
  % \label{figure:}
  \end{subfigure}%
  \begin{subfigure}{.5\textwidth}
    \centering
    \includegraphics[width=0.9\textwidth]{./images/tsneParametersTest/learningRate/lr200-1hDBSCAN.png}
    % \caption{}
    % \label{figure:}
  \end{subfigure}
	\caption{\textbf{1h} data files, t-SNE calculated with the following parameters: perplexity=40, n\_iter=5000, \textbf{learning\_rate=200}}
	\label{figure:1hlr200TSNE}
\end{figure}

% -- 3h, lr 200 --
\begin{figure}[H]
	\centering
	
  \centering
	\begin{subfigure}{.5\textwidth}
    \centering
    \includegraphics[width=0.9\textwidth]{./images/tsneParametersTest/learningRate/lr200-3hTSNE.png}
  % \caption{}
  % \label{figure:}
  \end{subfigure}%
  \begin{subfigure}{.5\textwidth}
    \centering
    \includegraphics[width=0.9\textwidth]{./images/tsneParametersTest/learningRate/lr200-3hDBSCAN.png}
    % \caption{}
    % \label{figure:}
	\end{subfigure}
	\caption{\textbf{3h} data files, t-SNE calculated with the following parameters: perplexity=40, n\_iter=5000, \textbf{learning\_rate=200}}
  \label{figure:3hlr200TSNE}
\end{figure}


%------------------ LEARNING RATE 400: ------------------
\subsubsection{Learning Rate = 400}
% -- 1h, lr 400 --
\begin{figure}[H]
  \centering
  \begin{subfigure}{.5\textwidth}
    \centering
    \includegraphics[width=0.9\textwidth]{./images/tsneParametersTest/learningRate/lr400-1hTSNE.png}
  % \caption{}
  % \label{figure:}
  \end{subfigure}%
  \begin{subfigure}{.5\textwidth}
    \centering
    \includegraphics[width=0.9\textwidth]{./images/tsneParametersTest/learningRate/lr400-1hDBSCAN.png}
    % \caption{}
    % \label{figure:}
  \end{subfigure}
	\caption{\textbf{1h} data files, t-SNE calculated with the following parameters: perplexity=40, n\_iter=5000, \textbf{learning\_rate=400}}
	\label{figure:1hlr400TSNE}
\end{figure}

% -- 3h, lr 400 --
\begin{figure}[H]
	\centering
	
  \centering
	\begin{subfigure}{.5\textwidth}
    \centering
    \includegraphics[width=0.9\textwidth]{./images/tsneParametersTest/learningRate/lr400-3hTSNE.png}
  % \caption{}
  % \label{figure:}
  \end{subfigure}%
  \begin{subfigure}{.5\textwidth}
    \centering
    \includegraphics[width=0.9\textwidth]{./images/tsneParametersTest/learningRate/lr400-3hDBSCAN.png}
    % \caption{}
    % \label{figure:}
	\end{subfigure}
	\caption{\textbf{3h} data files, t-SNE calculated with the following parameters: perplexity=40, n\_iter=5000, \textbf{learning\_rate=400}}
  \label{figure:3hlr400TSNE}
\end{figure}


%------------------ LEARNING RATE 600: ------------------
\subsubsection{Learning Rate = 600}
% -- 1h, lr 600 --
\begin{figure}[H]
  \centering
  \begin{subfigure}{.5\textwidth}
    \centering
    \includegraphics[width=0.9\textwidth]{./images/tsneParametersTest/learningRate/lr600-1hTSNE.png}
  % \caption{}
  % \label{figure:}
  \end{subfigure}%
  \begin{subfigure}{.5\textwidth}
    \centering
    \includegraphics[width=0.9\textwidth]{./images/tsneParametersTest/learningRate/lr600-1hDBSCAN.png}
    % \caption{}
    % \label{figure:}
  \end{subfigure}
	\caption{\textbf{1h} data files, t-SNE calculated with the following parameters: perplexity=40, n\_iter=5000, \textbf{learning\_rate=600}}
	\label{figure:1hlr600TSNE}
\end{figure}

% -- 3h, lr 600 --
\begin{figure}[H]
	\centering
	
  \centering
	\begin{subfigure}{.5\textwidth}
    \centering
    \includegraphics[width=0.9\textwidth]{./images/tsneParametersTest/learningRate/lr600-3hTSNE.png}
  % \caption{}
  % \label{figure:}
  \end{subfigure}%
  \begin{subfigure}{.5\textwidth}
    \centering
    \includegraphics[width=0.9\textwidth]{./images/tsneParametersTest/learningRate/lr600-3hDBSCAN.png}
    % \caption{}
    % \label{figure:}
	\end{subfigure}
	\caption{\textbf{3h} data files, t-SNE calculated with the following parameters: perplexity=40, n\_iter=5000, \textbf{learning\_rate=600}}
  \label{figure:3hlr600TSNE}
\end{figure}




%------------------ LEARNING RATE 800: ------------------
\subsubsection{Learning Rate = 800}
% -- 1h, lr 800 --
\begin{figure}[H]
  \centering
  \begin{subfigure}{.5\textwidth}
    \centering
    \includegraphics[width=0.9\textwidth]{./images/tsneParametersTest/learningRate/lr800-1hTSNE.png}
  % \caption{}
  % \label{figure:}
  \end{subfigure}%
  \begin{subfigure}{.5\textwidth}
    \centering
    \includegraphics[width=0.9\textwidth]{./images/tsneParametersTest/learningRate/lr800-1hDBSCAN.png}
    % \caption{}
    % \label{figure:}
  \end{subfigure}
	\caption{\textbf{1h} data files, t-SNE calculated with the following parameters: perplexity=40, n\_iter=5000, \textbf{learning\_rate=800}}
	\label{figure:1hlr800TSNE}
\end{figure}

% -- 3h, lr 800 --
\begin{figure}[H]
	\centering
	
  \centering
	\begin{subfigure}{.5\textwidth}
    \centering
    \includegraphics[width=0.9\textwidth]{./images/tsneParametersTest/learningRate/lr800-3hTSNE.png}
  % \caption{}
  % \label{figure:}
  \end{subfigure}%
  \begin{subfigure}{.5\textwidth}
    \centering
    \includegraphics[width=0.9\textwidth]{./images/tsneParametersTest/learningRate/lr800-3hDBSCAN.png}
    % \caption{}
    % \label{figure:}
	\end{subfigure}
	\caption{\textbf{3h} data files, t-SNE calculated with the following parameters: perplexity=40, n\_iter=5000, \textbf{learning\_rate=800}}
  \label{figure:3hlr800TSNE}
\end{figure}


%------------------ LEARNING RATE 1000: ------------------
\subsubsection{Learning Rate = 1000}
% -- 1h, lr 1000 --
\begin{figure}[H]
  \centering
  \begin{subfigure}{.5\textwidth}
    \centering
    \includegraphics[width=0.9\textwidth]{./images/tsneParametersTest/learningRate/lr1000-1hTSNE.png}
  % \caption{}
  % \label{figure:}
  \end{subfigure}%
  \begin{subfigure}{.5\textwidth}
    \centering
    \includegraphics[width=0.9\textwidth]{./images/tsneParametersTest/learningRate/lr1000-1hDBSCAN.png}
    % \caption{}
    % \label{figure:}
  \end{subfigure}
	\caption{\textbf{1h} data files, t-SNE calculated with the following parameters: perplexity=40, n\_iter=5000, \textbf{learning\_rate=1000}}
	\label{figure:1hlr1000TSNE}
\end{figure}

% -- 3h, lr 1000 --
\begin{figure}[H]
	\centering
	
  \centering
	\begin{subfigure}{.5\textwidth}
    \centering
    \includegraphics[width=0.9\textwidth]{./images/tsneParametersTest/learningRate/lr1000-3hTSNE.png}
  % \caption{}
  % \label{figure:}
  \end{subfigure}%
  \begin{subfigure}{.5\textwidth}
    \centering
    \includegraphics[width=0.9\textwidth]{./images/tsneParametersTest/learningRate/lr1000-3hDBSCAN.png}
    % \caption{}
    % \label{figure:}
	\end{subfigure}
	\caption{\textbf{3h} data files, t-SNE calculated with the following parameters: perplexity=40, n\_iter=5000, \textbf{learning\_rate=1000}}
  \label{figure:3hlr1000TSNE}
\end{figure}



\subsubsection{Learning Rate Detailed Comparison Results }
\label{appendix:comparelearningRateDetailed}

\begin{figure}
  \centering
  \includegraphics[width=0.8\textwidth]{./images/tsneParametersTest/learningRate/learningRateEvaluationScoresDetailed.png}
  \caption{Comparison of Silhouette Coefficient, Davies-Bouldin Index, and Caliński-Harabasz Index for different t-SNE \textbf{learning rate} values.}
  \label{figure:learningRateEvaluationScoresDetailed}
\end{figure}

\begin{figure}
  \centering
  \includegraphics[width=0.8\textwidth]{./images/tsneParametersTest/learningRate/learningRateEvaluationScoresDetailed2.png}
  \caption{Comparison of Silhouette Coefficient, Davies-Bouldin Index, and Caliński-Harabasz Index for different t-SNE \textbf{learning rate} values.}
  \label{figure:learningRateEvaluationScoresDetailed2}
\end{figure}

%.................................COMPARISON AVERAGES..................................
\subsubsection{Learning Rate Comparison Results (Average of two different t-SNE runs)}
\label{appendix:compareAverageLearningRate}


\begin{figure}[H]
  \centering
  \includegraphics[width=0.8\textwidth]{./images/tsneParametersTest/learningRate/learningRateEvaluationScoresAverage.png}
  \caption{Comparison of Silhouette Coefficient, Davies-Bouldin Index, and Caliński-Harabasz Index for different t-SNE \textbf{learning rate} values.}
  \label{figure:learningRateEvaluationScoresAverage}
\end{figure}

\begin{figure}[H]
  \centering
  \includegraphics[width=0.8\textwidth]{./images/tsneParametersTest/learningRate/learningRateEvaluationScoresAverageDetailed.png}
  \caption{Comparison of Silhouette Coefficient, Davies-Bouldin Index, and Caliński-Harabasz Index for different t-SNE \textbf{learning rate} values.}
  \label{figure:learningRateEvaluationScoresAverageDetailed}
\end{figure}

\begin{figure}[H]
  \centering
  \includegraphics[width=0.8\textwidth]{./images/tsneParametersTest/learningRate/learningRateEvaluationScoresAverageDetailed2.png}
  \caption{Comparison of Silhouette Coefficient, Davies-Bouldin Index, and Caliński-Harabasz Index for different t-SNE \textbf{learning rate} values.}
  \label{figure:learningRateEvaluationScoresAverageDetailed2}
\end{figure}

\begin{figure}[H]
  \centering
  \includegraphics[width=0.8\textwidth]{./images/tsneParametersTest/learningRate/learningRateEvaluationScoresAverageDetailed3.png}
  \caption{Comparison of Silhouette Coefficient, Davies-Bouldin Index, and Caliński-Harabasz Index for different t-SNE \textbf{learning rate} values.}
  \label{figure:learningRateEvaluationScoresAverageDetailed3}
\end{figure}






\subsubsection{Learning Rate Comparison of 20 and 800}
\label{appendig:compareLearningRate20and800}

\begin{figure}[H]
  \centering
  \includegraphics[width=1\textwidth]{./images/tsneParametersTest/learningRate/learningRateEvaluationScoresAverageDetailed4.png}
  \caption{Comparison of Silhouette Coefficient, Davies-Bouldin Index, and Caliński-Harabasz Index for the t-SNE \textbf{learning rate} values 20 and 80.}
  \label{figure:learningRateEvaluationScoresAverageDetailed4}
\end{figure}

%------------------ 1h: ------------------
\begin{figure}[H]
  \centering
  \begin{subfigure}{.5\textwidth}
    \centering
    \includegraphics[width=0.9\textwidth]{./images/tsneParametersTest/learningRate/lr201h-DBSCANCompare.png}
  % \caption{}
  % \label{figure:}
  \end{subfigure}%
  \begin{subfigure}{.5\textwidth}
    \centering
    \includegraphics[width=0.9\textwidth]{./images/tsneParametersTest/learningRate/lr8001h-DBSCANCompare.png}
    % \caption{}
    % \label{figure:}
  \end{subfigure}
	\caption{\textbf{1h} data files comparison of learning rate: a) 20, b) 800}
	\label{figure:1h-learningRateComparison20and800}
\end{figure}
%------------------ 3h: ------------------
\begin{figure}[H]
  \centering
  \begin{subfigure}{.5\textwidth}
    \centering
    \includegraphics[width=0.9\textwidth]{./images/tsneParametersTest/learningRate/lr203h-DBSCANCompare.png}
  % \caption{}
  % \label{figure:}
  \end{subfigure}%
  \begin{subfigure}{.5\textwidth}
    \centering
    \includegraphics[width=0.9\textwidth]{./images/tsneParametersTest/learningRate/lr8003h-DBSCANCompare.png}
    % \caption{}
    % \label{figure:}
  \end{subfigure}
	\caption{\textbf{3h} data files comparison of learning rate: a) 20, b) 800}
	\label{figure:3h-learningRateComparison20and800}
\end{figure}


\clearpage