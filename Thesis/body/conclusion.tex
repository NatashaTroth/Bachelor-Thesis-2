%(500 words)
% in conclusion

This thesis compared different time deltas for aggregation, to determine which one is ideal to construct high quality clusterings from smartphone sensor and usage data. The data was recorded from different test subjects for the SmartEater mobile health app. The datasets aggregated into 1h and 3h files were preprocessed, in which missing values, unnecessary columns and rows with more than 50\% zeros were removed. The resulting rows were normalised using z-score normalization. Using t-SNE, the 8 existing dimensions (attributes) were reduced to 2 and visualised in scatter plots. Such plots were created for each of the total 10 time lengths (1h and 3h files combined). DBSCAN and OPTICS clustering algorithms were used to group the data points together into clusters. The Silhouette Score, Davies-Bouldin Index, and Caliński-Harabasz Index are used to mathematically evaluate the resulting clusters for each time length. The comparison of these scores suggests, that the following time lengths produce the most distinct and well defined clusters: 2h and 1h (from the 3h dataset), and 1h and 30 min (from the 1h dataset). The results of this study suggest, that various time lengths might be necessary to receive the clearest clusters. User studies to evaluate hand drawn clusters would be a further step to identify appropriate time lengths to generate more distinct clusters and therefore be used to predict eating crises.
