%(500 words)
% in conclusion

This thesis compares different time deltas for aggregation, to determine which one is ideal to construct high quality clusterings from smartphone sensor and usage data. This data was recorded from different test subjects for the SmartEater mobile health app. The in 1h and 3h aggregated datasets were preprocessed, in which missing values, unnecessary columns and rows with more than 50\% zeros were removed. The resulting rows were normalised using z-score normalization. Using t-SNE, the 8 existing dimensions (attributes) were reduced to 2 and visualised in scatter plots. Such plots were created for each time length, a total of 10, within the dataset. DBSCAN and OPTICS clustering algorithms were used to group the data points together into clusters. The Silhouette Score, Davies-Bouldin Index and Caliński-Harabasz Index are used to mathematically evaluate the resulting clusters for each time length. The comparison of these scores leads to believe, that the following three time lengths produce the most distinct and well defined clusters:
30 minutes (from the 1h dataset), 1 hour, and 2 hours (both from the 3h dataset).




% The resulting dataset wa


\textbf{todo: finish}
