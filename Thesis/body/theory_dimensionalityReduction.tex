%https://www.amazon.de/dp/069107951X/ref=sr_1_1?dchild=1&keywords=9780691079516&qid=1584700659&sr=8-1
%  author={Bellman, R. and Rand Corporation and Karreman Mathematics Research Collection},
%https://books.google.at/books?hl=en&lr=&id=iwbWCgAAQBAJ&oi=fnd&pg=PR9&dq=Adaptive+Control+Processes&ots=bDJ9XsIa7c&sig=Rc5D_JAZc7a4FigIbt3DmiNYhJI#v=snippet&q=curse%20of%20dimensionality&f=false
\textcite{bellman1957dynamic}[20-22] first introduces the \textit{curse of dimensionality}. The curse effects a mathematical model, when there are a large number of variables. The real world is complex and by trying to incorporate as many real world features into a mathematical model as possible, it becomes complicated. A too simple model however, will not be suitable for prediction. 
\textcite{bellman1961adaptive}[94] further details the results of \textit{the curse of dimensionality}. Functions with one variable can be visualised as curve in a 2D space and a function with two variables in a 3D space. Depicting functions with more variables however, is more problematic (both for visualisation and tabulation). As stated by \textcite{han2011data}[93], dimensionality reduction is a data reduction method. Data reduction is utilised to attain a smaller, more concentrated dataset, whilst mostly keeping the integrity of the initial dataset. Principal components analysis and t-SNE are dimensionality reduction techniques used to reduce the SmartEater dataset.
%TODO: not sure if i've understood this next bit right:
% \textcite{bellman1961adaptive}[94, 198] further gives the following example: Imagine if the variables of a function take on the values between 1 and 100. While a function with one variable would need to tabulate 100 values, a function with 2 variables would need to tabulate 100 x 100 = 10\textsuperscript{4} values and a function with 3 variables 10\textsuperscript{6}. Each additional variable adds more complexity.
%...adds more complexity to the ?
% Multidimensional variational problems require more complex methods to solve them. 

% Imagine, for example, the variables are defined between 0 and 1 (e.g. u(t), where 0 $\leq$ t $\leq$ 1), at the points k(0.01). A function with one variable would require 100 values (the variable would take the numbers 0 to 100) to be tabulated. 





%TODO: FIND OUT HOW PRINCIPAL COMPONENTS ARE CALCULATED IN THE FIRST PLACE
% %TODO: FIND OUT WHO FIRST SUGGESTED IT FOR DATA MINING? ALTHOUGH DON'T THINK I'LL FIND MUCH
% According to \textcite{DataMiningAndPredictiveAnalytics}[92, 93], a high amount of predictor variables in data mining can lead to overfitting and overlooking crucial relationships between predictors. Dimensionality reduction techniques have the ability to reduce the number of predictor items, aid in ensuring that these predictor items are independent, and present a framework for interpretability of the results.





%Analysis of a complex of statistical variables by harold hotelling
%https://babel.hathitrust.org/cgi/pt?id=wu.89097139406&view=1up&seq=20
%On lines and planes of closest fit to systems of points in space
%-pdf
\subsubsection{Principal Components Analysis (PCA)}
Principal Components Analysis was first proposed by \textcite{OnLinesAndPlanes1901} and \textcite{hotelling1933analysis}.
Pearson's approach is to identify a line or plane that best fits the collected variables plotted to a plane. In order to determine the best fitting line or plane, means, standard-deviations, and correlations are used \autocite{OnLinesAndPlanes1901}[559-560].
\textcite{hotelling1933analysis}[5] introduces his method as the method of principal components. \textcite{jolliffe2002PCA}[7] clarifies, that while these two papers used different methods, the standard algebraic
derivation was announced by \textcite{hotelling1933analysis}. According to \textcite{han2011data}[95-96], the general idea is to calculate k orthonormal vectors, the so called principal components. These unit vectors present a basis for the input data, which are a linear combination of the principal components. \textcite{DataMiningAndPredictiveAnalytics}[94] explain, that the principal components can be discovered, by rotating the initial coordinate system to the direction of maximum variability. These then create a new coordinate system. \textcite{hotelling1933analysis}[4, 5, 15, 18] explains, that the components are chosen with decreasing amount of variance. Therefore, the one with the highest variance is chosen first. The next highest is chosen orthogonal to the one before, and so on. \textcite{han2011data}[93, 95-96] mentions, in data mining, the vectors with the lowest variance are removed, thus reducing the number of dimensions. Despite the loss of data, the components with higher variance can approximate the original data.

%TODO: NOT SURE WHAT n IS - PAGE 1 SAYS NUMBER OF VARIABLES, BUT NOT SURE IF THAT REALLY MAKES SENCE HERE - pretty sure its dimensions, mentioned in next paragraph:
% The next highest (\textit{$\gamma$\textsubscript{2}}) is chosen orthogonal to \textit{$\gamma$\textsubscript{1}} and so on, until the number \textit{n} dimensions are reached (\textit{$\gamma$\textsubscript{n}}). The components left with small variance are disregarded, since they are trivial. 
% The method results in selecting a new set of coordinate axes which correspond to the principal axes of ellipsoids. 
%NOT SURE IF UNDERSTOOD THIS RIGHT, END PAGE 24&25:
% In an example, Helling selects the two of four calculated principal components with the highest variance, therefore being able to display the data on a two dimensional scatter diagram. (?)
%10
% Geometrically, the new coordinate axes are rotated to lie along the principal axes of the ellipsoids.





% \textbf{!!!TODO: IF USE PCA, GO MORE INTO DETAIL AND TAKE IT FROM jolliffe2002PCA (=pdf) INSTEAD OF han2011data}.
% \textbf{todo: more detail (if use PCA in experiment)}.


% \textcite{han2011data}[95-96]
% \begin{enumerate}
%   \item The first step is to standardise the input data, therefore making the data-range identical. \textcite{DataMiningAndPredictiveAnalytics}[94] declares, that after standardising the data, the mean is zero and the standard deviation is one.
%   \item Next, k orthonormal vectors are calculated, the so called \textit{principal components}. These unit vectors present a basis for the input data, which are a linear combination of the principal components. \textcite{DataMiningAndPredictiveAnalytics}[94] explain, that the principal components can be discovered, by rotating the initial coordinate system to the direction of maximum variability. These then create a new coordinate system. 
%   \item In the following step, as stated by \textcite{han2011data}[95-96], the principal components are put into order by their decreasing significance/strength, thus presenting their variance. These vectors are used as new axes for the data, the first axis exhibits the highest variance.
%   \item Due to the decreasing order  of variance, the vectors with the lowest variance can be removed, therefore reducing the amount of data and number of dimensions. Despite the loss of data, the components with higher variance can approximate the original data.
% \end{enumerate}

%TODO: EXPLAIN MORE IN DETAIL!! ALTHOUGH - MIGHT BE DONE IN PRACTICAL PART - ASK

%TODO: explain what variance is, maybe also skewed data
%variance = range, want data as spread out as we can find it
%either, find highest variance (where points are spread out the most), or minimum error(distance to axis)
%retain most of the information
%since eigenvalues - each principal component ends up orthogonal
%youtuber video: which direction will you pick, that maximizes the variance
%page 93



\subsubsection{t-Distributed Stochastic Neighbor Embedding (t-SNE)}
\label{section:tSNE}
In their paper, \textcite{maaten2008visualizing}[2579-2580] introduce t-SNE, a method of dimensionality reduction and visualising high dimensional data on a two or three-dimensional plot. The authors mention, that while linear methods like PCA concentrate on making sure low-dimensional representations of data points are apart from each other, they are typically unable to bring similar low-dimensional representations near. 

The authors describe, that t-SNE is a variation of Stochastic Neighbor Embedding (SNE), introduced by \textcite{hinton2003stochastic}. \textcite{maaten2008visualizing}[2581] explain, that the first step in SNE is to transform the Euclidean distances between points in high-dimensional space into the conditional similarity probabilities. According to \textcite{hinton2003stochastic}[2], such a probability \textit{p\textsubscript{i,j}} would arise from the similarity between the two data points \textit{x\textsubscript{i}} and \textit{x\textsubscript{j}}. This is the probability of \textit{x\textsubscript{i}} picking the point \textit{x\textsubscript{j}} as its neighbour. \textcite{maaten2008visualizing}[2581] clarify, that this is the probability of \textit{x\textsubscript{i}} picking the point \textit{x\textsubscript{j}} as its neighbour under the circumstance, that the neighbours were chosen according to their probability density under a Gaussian, \textit{x\textsubscript{i}} being its centre. This results in the conditional probability being high for close data points and imperceptibly small for those further apart.
In a similar way, the conditional probability between the low-dimensional data points \textit{y\textsubscript{i}} and \textit{y\textsubscript{j}} (counterparts to high-dimensional \textit{x\textsubscript{i}} and \textit{x\textsubscript{j}}), represented by \textit{q\textsubscript{i,j}}, is calculated. 
\textcite{maaten2008visualizing}[2581] state, that if the similarity of \textit{x\textsubscript{i}} and \textit{x\textsubscript{j}} has been correctly mapped by \textit{y\textsubscript{i}} and \textit{y\textsubscript{j}}, then \textit{p\textsubscript{i,j}} will be equal to \textit{q\textsubscript{i,j}}. 


\textcite{hinton2003stochastic}[2] further explain, that the goal is to reduce the difference between \textit{p\textsubscript{i,j}} and \textit{q\textsubscript{i,j}}, thus finding a suitable low-dimensional representation of the high-dimensional data. This is accomplished by minimising a cost function, comprised of the sum of the Kullback-Leibler divergences between {p\textsubscript{i,j}} and \textit{q\textsubscript{i,j}}. 
\textcite{maaten2008visualizing}[2583] then list the advantages of t-SNE over SNE are the improvement of its cost function (symmetrised version of SNE) and the use of Student-t distribution as opposed to Gaussian, for the similarity calculation in the low-dimensional area.
%TODO!!! FINISH

To apply t-SNE to a dataset, \textcite{maaten2008visualizing}[2582] describe, the $\sigma$\textsubscript{i} has to be chosen for variance of the Gaussian centred over \textit{x\textsubscript{i}}. Depending on the density, different values of $\sigma$\textsubscript{i} provide better results. For example, in a denser region, a smaller $\sigma$\textsubscript{i} is recommended, but not in less denser regions. This makes it unlikely to be able to find an overall optimal  $\sigma$\textsubscript{i} for the dataset. \textcite{hinton2003stochastic}[2] declare, that $\sigma$\textsubscript{i} can be chosen by hand or is found using a binary search. This would equalise the distribution entropy to log\textit{k}, \textit{k} being the effective number of local neighbours or the so called perplexity. The perplexity is a parameter chosen by the user for the SNE by hand. \textcite{maaten2008visualizing}[2582] imply, that this parameter should be chosen between 5 and 50 and that SNE is quite robust to the chosen value. 
Another parameter to be set is the learning rate. \textcite{han2011data}[332] clarify, that the learning rate parameter is used to support the finding of global minimums, instead of being caught in a local minimum. A learning rate set too low will result in slow learning. In contrast, a learning rate set too high could lead to poor results.




% @article{tsneAutomaticPerplexity,
%   author    = {Yanshuai Cao and             Luyu Wang},
%   title     = {Automatic Selection of t-SNE Perplexity},
%   journal   = {CoRR},
%   volume    = {abs/1708.03229},
%   year      = {2017},
%   url       = {http://arxiv.org/abs/1708.03229},
%   archivePrefix = {arXiv},
%   eprint    = {1708.03229},
%   timestamp = {Mon, 13 Aug 2018 16:47:03 +0200},
%   biburl    = {https://dblp.org/rec/journals/corr/abs-1708-03229.bib},
%   bibsource = {dblp computer science bibliography, https://dblp.org}
% }

% \textcite{tsneAutomaticPerplexity}[1-2] point out, that this parameter has to be chosen by visually comparing the t-SNE outcomes. The authors propose a method to automatically select the t-SNE parameter perplexity. This parameter influences the results received from t-SNE and is difficult to optimise for inexperienced users. The authors explain, that the Kullback-Leibler (KL) divergences received from different complexities cannot solely be compared to find the most suitable perplexity. As the perplexity rises, the KL falls, selecting the smallest KL would therefore result in a very high perplexity (e.g. 400), whereas \textcite{maaten2008visualizing}[2582] recommend setting the perplexity to between 5 and 50. \textcite{tsneAutomaticPerplexity}[2-] further explains, that when the perplexity set to the number of data points, no interesting structure is found, only a Gaussian or uniform like blob. Accordingly, it could be perceived, that a trade off between the final Kullback-Leibner divergence and the Perplexity (\textit{Perp}) might result in good t-SNE results. The authors propose the following formula:

% \[
%   S(Perp) = KL(P||Q) + log(n)\frac{Perp}{n}
% \]



% would be the conditional probability, that the point \textit{x\textsubscript{i}} would pick the point \textit{x\textsubscript{j}} as its' neighbour.


% The similarity
% of datapoint x j to datapoint xi is the conditional probability, p jji, that xi would pick x j as its neighbor
% if neighbors were picked in proportion to their probability density under a Gaussian centered at xi.



% Notes from https://www.youtube.com/watch?v=RJVL80Gg3lA :
% Represent each highdimensional data by a point
% similar data by close points
% dissimilar data by points further away
% embed points in low dimensional map - visualise as scattterplot

% distances in the low d map should reflect the similarities in the high d map

% discrepancy: an illogical or surprising lack of compatibility or similarity between two or more facts.pca -  concerned with preserving large pairwise distances in the map - but are such distances very reliable?

% tsne: in hd space- measure similarities between points.
% gaussian over point and measure density of all other points under it and then renormalize
%  - like probability of points where probability is proportional to similarity of the points
%  - We set the bandwith sigmai such that the conditional has a fixed perplexity (fixed number of points in gaussian, to account for denser regions)

%  then do the same for low dim space (qij) - if qij and pij are similar - then good mapping - qij should reflect pijs as well as possible
 
%  layout qijs values in map so that as similar as possible to pij - move points around, so that kl kullback-leibler divergence becomes small

%  2 hd points similar (close together) - large pij value - have to make sure also gets large qij value