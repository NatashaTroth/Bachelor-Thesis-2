The resulting clusters received from the previously mentioned clustering algorithms are assessed in the \textit{cluster evaluation} step. \textcite{han2011data}[396-..] describe this stage as assessing the quality of the results.
There are different steps to be taken in evaluating clusters. 
\begin{itemize}
  \item \textbf{Assessment of the cluster tendency}:
  The tendency must be assessed, meaning it is tested, whether structures exist that aren't random. Running a clustering algorithm on any data set will return clusters. However, only nonrandom structures are significant and not misleading. For example, if a data set consists of data points that are uniformly distributed, if a clustering algorithm delivers clusters, these will be random and have no purpose. Spatial randomness tests (e.g. Hopkins Statistic) can be used to measure how likely the data was created by uniform data distribution.

  \item \textbf{Establishing the number of clusters}:
  Next, the number of clusters found in the data set needs to be established. For some clustering methods (e.g. \textit{k}-means), this number is defined before the clustering process. THERE IS MORE HERE - see page 398

  \item \textbf{Evaluation of the cluster quality}:
  Finally, the cluster quality needs to be evaluated......
\end{itemize}



%TODO: this section, then also the section on high dimensional clustering (p. 414). Then maybe also outlier detection p445, and "Data mining trends and research frontiers" p481