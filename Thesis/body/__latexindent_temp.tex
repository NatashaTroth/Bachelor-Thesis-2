% (500 words)
%LEITFADEN:
% Die Einleitung dient der Einführung in das Thema, indem eine spezifische Fragestellung (For- schungsfrage) herausgearbeitet, plausibel formuliert, differenziert und begründet wird. Die methodische Vorgehensweise („Wie wird das Thema bearbeitet?“) muss begründet und auf das Thema und die Frage- und Problemstellung der Arbeit bezogen werden. Des Weiteren sollte in der Einleitung oder in einem gesonderten Kapitel (Verwandte Arbeiten) die zur Bearbeitung herangezogene wissenschaftliche Literatur sowie die spezifische Auswahl von Quellen und Texten dargestellt und erläutert werden.





%TODO: also add a paper on eating crises...
% the book
\textcite{han2011data}[18, 32, 362, 363, 367] declare, that data mining is used to discover patterns and knowledge from data.
% (17) This includes cleaning data, combining multiple sources, selecting and transforming relevant data, and extracting and evaluating data patterns.
Cluster Analysis is a type of machine learning algorithm known as unsupervised machine learning. It is used in data mining to divide data into groups (clusters). Each cluster contains data that is similar to each other, but dissimilar to the data allocated to other clusters. Cluster Analysis can be used to acquire knowledge on the distribution of the data, discover characteristics, detect outliers and reduce noise, or to pre-process data for other algorithms. 


%Many clustering algorithms determine clusters based on Euclidean or Manhattan distance measures
%book - page 366-368, 373, 374, 385, 392, 414
There are several different methods to create clustering. \textcite{han2011data}[362, 364, 366-367, 385, 392] explain, that objects are often arranged into clusters using distance measures (e.g. Euclidean or Manhatten distance measures). 
% The authors divide the clustering algorithms into the following categories:
% \begin{itemize}
% 	\item Partitioning methods (examples: k-means, k-medoids)
% 	\item Hierarchical methods (examples: BIRCH, Chameleon)
% 	\item Density-based methods (examples: DBSCAN, OPTICS)
% 	\item Grid-based methods (examples: STING, CLIQUE)
% \end{itemize}
%end book



%Related work
%Evidence Analysis to Basis of Clustering
\textcite{forensics} introduce in their paper, how clustering can be used in digital forensics to provide information on all the events that led up to a certain crime. They used ascending hierarchical clustering to receive clusters of events (e.g. phone calls, SMS) ordered in time, thus creating a timeline of events leading up to the incident.
% The idea was by using a hierarchical fusion process, it would be possible to create a chain of events 

%Convex-Hull & DBSCAN Clustering to Predict Future Weather
% Another example where clustering was implemented was by /textcite. 
\textcite{convexhullDbscan}[1,2,6,7] give an example, where clustering was implemented to predict future weather. Air pollutant data was preprocessed and then arranged into clusters using (incremental) DBSCAN clustering. Finally, priority based protocol was used on them to predict weather conditions and a temperature range. The accuracy of the technique, based on hit and miss times, was calculated to approximately 74.5\%.

%end related work

% Talk about SmartEater.
SmartEater \footnote{\url{https://sites.google.com/site/eatingandanxietylab/resources/smarteater}} is an upcoming mHealth (mobile health) app, with the goal to provide the user with content-dependent feedback, to avert a food craving episode. The app will predict future eating crises based on the user's past behaviour. In order to reduce intense user input, the app records and uses various smartphone sensor data.  With the help of data mining, machine learning algorithms, and pattern recognition, this recorded situational context data will aid in predicting stress. The following data is recorded by the app:

\begin{enumerate}
	\item Background volume
	\item Relative movement of the smartphone (gyro and accel)
	\item Time and duration of phone calls (without storing the numbers)
	\item Time of messages (e.g. SMS, WhatsApp) (without collecting identifying information such as content, addresses, numbers)
	\item Screen activity (so-called touch events)
	\item Screen-on-time (illuminated display)
	\item Ambient brightness
	\item Data volume per unit of time (summary value of all smartphone activities on the internet)
	\item Switch-on and switch-off times of the smartphone
\end{enumerate}


%cite smarteater website  - as footnote
This sensor data will be recorded for different lengths of time. It is necessary to establish which time period will be most fitting to make accurate predictions for the future. This thesis will use cluster analysis to determine which time period is most significant.

According to \textcite{han2011data}[414], the above-mentioned clustering methods work well with data sets that are not high-dimensional and have less than 10 attributes. Since the SmartEater data set only has 9 dimensions, it is not considered high-dimensional. This paper will therefore utilise these clustering methods. Since different clustering algorithms can yield different results, multiple methods will be used and compared.

%book page 93
To reduce the size and amount of data, dimensionality reduction will be used. \textcite{han2011data}[93] define dimensionality reduction as a type of data reduction, which removes random attributes and creates a smaller data set with close to equal integrity. This thesis will use principal component analysis (PCA) to reduce the dimensionality.
%Visualizing Data using t-SNE p2579 (abstract!)
Furthermore, T-Distributed Stochastic Neighbor Embedding (t-SNE) will be employed to depict the data set in this thesis. \textcite{maaten2008visualizing}[2579] first introduce t-SNE, which is used to visualise data with a higher dimensionality. 

The clustering methods will be implemented using a Python machine learning platform or library (e.g. Anaconda\footnote{\url{https://www.anaconda.com/}}, scikit-learn\footnote{\url{https://scikit-learn.org/stable/}}). Next, these will be implemented on the other time lengths. The resulting clusters of each time length will be compared to one another and evaluated. 
\textcite{rousseeuw1987silhouettes} reveals how silhouettes can be used to measure the separation between clusters and therefore evaluate good resulting are.

% Evaluation examples given by \textcite{han2011data}[396-399] include clustering tendency, intrinsic and extrinsic measurements. 

% (e.g. Silhouette Coefficient \autocite{kaufman2009finding}[87])

The thesis will be structured as follows: The first section will briefly present existing work relating to this subject. The following chapter will concentrate on the theory of data mining and cluster analysis. After covering these topics, the next section will describe the conducted experiment and its results. In the final sections, the findings of the experiment will be discussed and summarised. 


!!!WRITE ABOUT EATING DISORDERS.., ALSO WRITE ABOUT MOBILE HEALTH APPS
%Snoring paper abstract: Mobile Health broadens the accessibility to healthcare applications using mobile devices, such as smartphones.