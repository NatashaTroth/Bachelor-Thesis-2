%(500 words)
% in conclusion

This thesis compared different time deltas for aggregation, to determine which one is ideal to construct high quality clusterings from smartphone sensor and usage data. The data was recorded from different test subjects for the SmartEater mobile health app. The datasets aggregated into 1h and 3h files were preprocessed, in which missing values, unnecessary columns and rows with more than 50\% zeros were removed. The resulting rows were normalised using z-score normalization. Using t-SNE, the 8 existing dimensions (attributes) were reduced to 2 and visualised in scatter plots. Such plots were created for each of the total 10 time lengths (1h and 3h files combined). DBSCAN and OPTICS clustering algorithms were used to group the data points together into clusters. The Silhouette Score, Davies-Bouldin Index, and Caliński-Harabasz Index are used to mathematically evaluate the resulting clusters for each time length. The comparison of these scores leads to believe, that the following three time lengths produce the most distinct and well defined clusters:
30 minutes (from the 1h dataset), 1 hour, and 2 hours (both from the 3h dataset). User studies to evaluate hand drawn clusters would be further step to solidify the results.


% It might also be beneficial to support the mathematical evaluations with hand drawn clusters found by test users in user studies. 

\textbf{DAD: do you think something else is missing here? I'm thinking maybe to also explain a bit of the discussion here, but not really sure what is the most important thing.}