The resulting clusters received from the previously mentioned clustering algorithms are assessed in the \textit{cluster evaluation} step. \textcite{han2011data}[396] describe this stage as assessing the quality of the results.
There are different steps to be taken in evaluating clusters. 
 
  \subsubsection{Assessment of the cluster tendency}
  As explained by \textcite{han2011data}[396-397], the tendency must be assessed, meaning it is tested, whether structures exist that aren't random. Running a clustering algorithm on any data set will return clusters. However, only non random structures are significant and not misleading. For example, if a data set consists of data points that are uniformly distributed, if a clustering algorithm delivers clusters, these will be random and have no purpose. Spatial randomness tests (e.g. Hopkins Statistic) can be used to measure how likely the data was created by uniform data distribution.

 

  \subsubsection{Evaluation of the cluster quality}
  (\textbf{Todo: adjust to the methods used in the experiment})

  \textcite{han2011data}[399, 401] further mentions, that the cluster quality needs to be evaluated. Generally, there are two ways to measure the quality of clustering: extrinsic methods and intrinsic methods. In extrinsic methods, there is a ground truth available, therefore these are also referred to as supervised methods. This ground truth is usually produced by experts (humans). Intrinsic methods are used, when there is no ground truth available. In intrinsic methods, the clusters are evaluated by how well they are separated from one another and how compact they are (e.g. \textit{silhouette coefficient}).
  % NOT SURE IF SHOULD EXPLAIN EXTRINSIC, SINCE DON'T THINK USING.
  The experiment described in this paper uses the intrinsic method silhouette coefficient, since there is no ground truth for comparison.
  
  
  \paragraph{Silhouette Coefficient}
  In his paper, \textcite{rousseeuw1987silhouettes}[53-57, 59] proposes a new graphical display using silhouettes, to help determine how well objects belong to their assigned clusters. It can be used to interpret and validate the results of clustering. It is also utilised to compare the resulting clusters with those output using alternative algorithms (the input data being the same). In an example, where countries are assigned a value of how dissimilar they are to another country, the results are listed in a table. A structure contained in the results (consisting of 66 numbers), is hard to identify. Therefore, the countries are categorised into clusters using k-median. It is however uncertain, whether the clusters follow a specific structure, or if the groups are simply artificial. With the use of silhouettes, the author's goal is to answer the following questions: Is the quality of the clusters high, therefore the dissimilarities of the objects within a cluster are small, and large compared to the objects in other clusters? Are the objects well-classified, misclassified and which ones were not classified (between clusters)? Is it possible to perceive which "natural" clusters are available in the data set? \\
  According to Rousseeuw, the silhouettes are ideal when the distance between objects are on a ratio scale (e.g. Euclidean distances) and when the goal is to receive clear and compact clusters. For each object \textit{i} (in cluster \textit{A}) the value \textit{s(i)} is calculated. \textit{a(i)} contains the average dissimilarity ob the object \textit{i} to each other object in the same cluster. If there are no other objects in the cluster, \textit{s(i)} is set to zero (most neutral value). \textit{b(i)} is determined, by firstly calculating the average dissimilarity, for each neighbouring cluster that isn't \textit{A}. The shortest of these values, therefore the next closest cluster to \textit{A}, is then assigned to \textit{b(i)}. This cluster can so to say be seen as the next best choice for \textit{i}. \textit{b(i)} can only be calculated, if there are other clusters beside \textit{A}.
  The formula is as follows:
  \[
    s(i) = \frac{b(i) - a(i)}{max\{a(i), b(i)\}}  
  \]
  The resulting value \textit{s(i)} is a number in the range of \textit{-1 $\leq$ \textit{s(i)} $\leq$ 1}:

  \[ s(i) =
  \begin{cases}
    1 - a(i)/b(i)       & \quad \text{if } a(i) < b(i)\\
    0       & \quad \text{if } a(i) = b(i)\\
 b(i)/a(i) - 1      & \quad \text{if } a(i) > b(i)

  \end{cases}
\]
A \textit{s(i)} value close to 1 reveals, that the dissimilarity within a cluster is smaller than the dissimilarity to the neighbouring cluster. Therefore it suggests, that the assignment of that object is good, since it is the most likely the most suitable cluster for \textit{i} (well-classified). A \textit{s(i)} value close to 0 means that \textit{a(i)} and \textit{b(i)} are almost equal and it is unclear whether \textit{A} or the neighbouring cluster is a more suitable fit. If \textit{s(i)} is close to -1, then the dissimilarity within a cluster is larger than the dissimilarity to the neighbouring cluster. Thus, it would have been more natural to assign \textit{i} to the neighbouring cluster, since it is closer to it (misclassified). The function can also be adapted to work with similarities.
The \textit{average silhouette width} for each cluster is received by calculating the average of all objects that belong said cluster. The higher the average silhouette width, the more pronounced the cluster is. An average score can also be calculated from each object \textit{i} for the entire plot (data set), the so called \textit{overall average silhouette width}. 

\textcite{silhouetteRelocatingMeasure}[15] use the silhouette coefficient in their proposed clustering algorithm, which clusters categorical data. The coefficient was used assess the quality of the clusters and to relocate objects to more fitting clusters. The cluster efficiency in their algorithm was therefore enhanced.


\subsubsection{Establishing the number of clusters}
 \textcite{han2011data}[398] states, that the number of clusters (\textit{k}) found in the data set needs to be established. For some clustering methods (e.g. \textit{k}-means), this number is defined before the clustering process. This number can be challenging to determine and depends on the shape and scale of the input data. A good number of clusters creates a balance between \textit{compressibility} and \textit{accuracy}. Having only one cluster would have maximum compression, but no value. Contrarily, if each data object formed its own cluster, the clusters would be most accurate, but not allow for summarisation of the data. 
\textcite{rousseeuw1987silhouettes}[59] describes, how silhouettes can be used to determine the ideal amount of clusters. Picture a data set with dense clusters which each have large distances to the other clusters. When \textit{k} is chosen too small, naturally occurring clusters must be artificially joined, to satisfy the value \textit{k}. Implementing the silhouette calculation will result in high within-cluster dissimilarities (\textit{a(i)}) leading to a narrow silhouette (small \textit{s(i)}). Likewise, if \textit{k} is chosen too large, natural clusters will have to be artificially split, in order to gain \textit{k} clusters. The objects in a split natural cluster will however still be very close to the other half of their cluster, therefore resulting in low dissimilarities between clusters (\textit{b(i)}) and a small \textit{s(i)}.
This logic denotes, that silhouettes should be capable of finding the most 'natural' number of clusters in a data set.


  
% %TODO: there is some more about it with examples - but not sure if need - from p. 584

\paragraph{Davies-Bouldin Index}



\paragraph{Calinski-Harabasz Index}
\textcite{calinskiHarabasz} introduce their
