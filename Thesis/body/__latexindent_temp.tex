%(4000 words)
% in experiment

%TODO: there are some interesting silhouette and cluster evaluation examples in rousseeuw \textcite{rousseeuw1987silhouettes} from page 60 (page 8 on the pdf)

%TODO: MAYBE EXPLAIN ACCELEROMETER AND GYROSCOPE - SEE PAGE 7 OF Automatic Annotation of Unlabeled Data from Smartphone-Based Motion and Location Sensors - pdf in related works folder

% \subsubsection{Data structure}
% Folder Structure:
% 	"aggregated": contains data aggregated for each user (= filename)
% 		- "1h": Data aggregated in 2.5h intervals. Each row is an aggregation of 1h in the past in 4 15min lags.
% 		- "3h": Data aggregated in 1.5h intervals. Each row is an aggregation of 3h in the past in 6 30min lags.

% 	"clusters": contains the cluster index for each data row of the aggregated data. Clusters were automatically detected by reducing the dimensionality of the aggregated data rows to 2 dimensions using t-SNE and then applying simple spectral clustering with k = 3.

% Columns:
% 	TIME: 			The timestamp of where the data is aggregated from (in format YYYY-DD-MM hh:mm:ss)
% 	ACC (1-N): 		Accelerometer values (we calculate the "average jerk"; N averages of lag-interval minutes over the aggregation interval in the past)
% 	AUDIO (1-N): 		Audio volume (same lagged averages as Accelerometer)
% 	SCRN (1-N): 		Percentage of Screen being on in lag-interval minutes (same lagged intervals as Accelerometer)
% 	NOTIF (1-N): 		Notification amount in lag-interval minutes (same lagged intervals as Accelerometer)
% 	LIGHT (1-N): 		Light Sensor values (same lagged intervals as Accelerometer)
% 	APP_COM (1-N):		App usage of category 'Communication' in percent of lag-interval minutes (i.e. if apps of this category were used for 5 minutes in the past 15 minutes the value is 0.66666)
% 	APP_VID (1-N):		App usage of category 'Video_Players', same principle as APP_COM
% 	APP_OTHER (1-N):	App usage of all other categories (excluding 'Video_Players' and 'Communication')



	