%DataMiningAndPredictiveAnalytics
%TODO: talk about why data mining and clustering is now so important, since so much data - also mentioned in data mining concepts.. book on page 363
%THERE ARE CASE STUDIES AT THE END OF THIS BOOK
\textcite{DataMiningAndPredictiveAnalytics}[4] declare that data mining is used to recognise patterns and trends in large amounts of data.

\textcite{han2011data}[16, 17] explain, that the term "data mining" is a misnomer. A more suitable phrase would be "knowledge mining from data". The word "mining" represents valuable nuggets found within large amounts of raw material. Other names used to describe the same process include: knowledge discovery from data (KDD), knowledge extraction, data/pattern analysis, data archaeology, and data dredging.
According to the authors, the discovery of data is an iterative process represented in the following steps:

\begin{enumerate}
  \item Data cleaning
  \item Data integration (combine multiple data sources)
  \item Data selection (relevant data is extracted)
  \item Data transformation (into applicable forms for data mining )
  \item Data mining (discover patterns)
  \item Pattern evaluation (determine if patterns have a meaning)
  \item Knowledge presentation
\end{enumerate}

%page 9 - there are other fallacies on this page that may be interesting
\textcite{DataMiningAndPredictiveAnalytics}[9-13, 15-16] Data mining requires continuous human supervision for quality monitoring and evaluation. Software alone will serve wrong results. Data mining is used for description of patterns and trends, estimation of numerical values, prediction of future results, classification of categorical variables, clustering of similar objects and association of attributes.

\textcite{han2011data}[18] list the following data forms, which are typically used for mining: database data, data warehouse data, and transactional data. Other forms include data streams, ordered/sequence data, graph or networked data, spatial data, text data, multimedia data, and the World Wide Web.



%page 17 - introduction into R



%page 160-163
\textcite{DataMiningAndPredictiveAnalytics}[160-163] describe the two types of data mining methods: \textit{supervised} and \textit{unsupervised}. The majority of methods are supervised. In supervised methods, there is a predefined target variable. The method receives several examples, where the target variable value is defined, thus learning which values of the target variable correspond to which values of the predictor variable. The goal of the unsupervised approach is to find patterns and structure in the inserted variables. Therefore, no target variable is established. Clustering is the most known unsupervised method.

\textcite{han2011data}[32, 363] describe supervised learning as \textit{learning by examples}, whearas unsupervised learning is \textit{learning by observation}. Using unsupervised machine learning, it is possible to detect classes within data.

%other book (data mining and predictive analysis)
As stated in \textcite{DataMiningAndPredictiveAnalytics}[160-163], problems that can occur in data mining methods are data dredging and overfitting. Data dredging is when false results arise in data mining due to random variations of data. Cross-validation is used to prevent data dredging, by guaranteeing that the results can be generalised to an independent data set. %there is more info about this on page 161
%WHAT ABOUT UNDER FITTING
Overfitting arises, when the provisional model tries to fit perfectly to the training model, thus leading to the accuracy being higher on the training set than on the test set.
EXPLAIN UNDERFITTING

\textcite{DataMiningAndPredictiveAnalytics}[164-165] describe another way to describe the overfitting/underfitting problem is through the bias-variance trade-off. Imagine a scatter plot with data points in two different colors, which need to be separated by a line. A low-complexity separator (e.g. a straight line) may have some classification errors (\textit{high bias}), however it needn't change much to accommodate new data points. Therefore, it has \textit{low variance}. A high-complexity separator (e.g. curvy line that can separate more the points correctly) reduces the amount of errors (\textit{low bias}), but has to change a lot when new data points are added. Thus, it has \textit{high variance}. The higher the complexity of the model gets, the bias is reduced, the variance however increases. The ideal model has neither high bias or variance.
%TODO: MAYBE ADD IMAGES - GOOD ONES ON THIS PAGE  









%BOOK Data mining concepts and techniques
%page 12 and 13 explains how much data exists, goes through google, basically why we need data mining


%page 18 - another sentence explaining what data mining is, but similar to one from other book
%page 18


%page 28, 29 - outlier analysis

\textcite{dataPreparationForDataMining}[71-72] describes outliers as objects that have low recurrence and separated from the main collection of values. These values are often mistakes and can lead to distortion of the data set. Insurance companies provide a good example of outliers. The majority of insurance claims are only for a small sum, however every so often a customer may be in need of a large claim.
\textcite{han2011data}[28-29]
Outliers are objects that vary to the general behaviour or model of the data. In some cases, the uncommon events are of more interest. One of these instances is detecting unusually large payments compared to the card holders normal payments, to uncover fraudulent usage of credit cards.

%page 32 - there is an example for unsupervised m. l. on this page, but not really sure if need it


%page 416
\textcite{han2011data}[416]
The clustering methods mentioned in section \ref{section:TheoryClusterAnalysis} have a good functionality when used on a dataset with fewer than 10 attributes.
Other ways to cluster high-dimensional data include \textit{subspace clustering}.Subspaces (subset of attributes) are investigated to find clusters. The CLIQUE method is used for subspace clustering.



%TODO: EXPLAIN HOW TO NORMALIZE - PAGE 105 OF DATA MINING CONCEPTS.. BOOK -> also explained in data mining and predictive analysis - page 94

%TODO:  data mining concepts and techniques: outlier detection p445, and "Data mining trends and research frontiers" p481