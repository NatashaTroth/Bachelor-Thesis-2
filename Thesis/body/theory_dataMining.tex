%DataMiningAndPredictiveAnalytics

\textcite{DataMiningAndPredictiveAnalytics}[4] declare, that data mining is used to recognise patterns and trends in large amounts of data. \textcite{han2011data}[16-18] explain, that the term "data mining" is a misnomer. A more suitable phrase would be "knowledge mining from data". The word "mining" represents valuable nuggets found within large amounts of raw material. Other names used to describe the same process include: knowledge discovery from data (KDD), knowledge extraction, data/pattern analysis, data archaeology, and data dredging. The discovery of data is an iterative process represented in the following steps: data cleaning, data integration (combine multiple data sources), data selection (relevant data is extracted), data transformation (into applicable forms for data mining), data mining (discover patterns), pattern evaluation (determine if patterns have a meaning), and knowledge presentation. 
%The following data forms, are typically used for mining: database data, data warehouse data, and transactional data. Other forms include data streams, ordered/sequence data, graph or networked data, spatial data, text data, multimedia data, and the World Wide Web. 
As stated by \textcite{DataMiningAndPredictiveAnalytics}[9-13, 15-16], data mining requires continuous human supervision for quality monitoring and evaluation. Software alone will serve wrong results. Data mining is used for description of patterns and trends, estimation of numerical values, prediction of future results, classification of categorical variables, clustering of similar objects and association of attributes. 

\textcite{DataMiningAndPredictiveAnalytics}[160] describe the two types of data mining methods: \textit{supervised} and \textit{unsupervised}. \textcite{han2011data}[363] interpret supervised learning as \textit{learning by examples}, whereas unsupervised learning is \textit{learning by observation}. \textcite{DataMiningAndPredictiveAnalytics}[160-163] continue, in supervised methods, there is a predefined target variable and the model learns which values of the target variable correspond to which values of the predictor variable. The goal of the unsupervised approach is to find patterns and structure in the inserted variables. Therefore, no target variable is established. Clustering is used in this thesis and further detailed in section \ref{section:TheoryClusterAnalysis}.