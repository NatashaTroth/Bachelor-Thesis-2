%data mining concepts..
\textcite{han2011data}[361-363] states that cluster analysis, or clustering, is used to group together objects similar to one another into a cluster. It therefore divides a data set of objects into subsets (clusters). The objects placed into one cluster are dissimilar to the objects assigned to other clusters. Therefore, such a cluster can also be defined as an implicit class. For this reason, clustering is occasionally referred to as automatic classification. The fact, that cluster analysis can find groups by itself, gives it its unique advantage. Clustering is a type of unsupervised machine learning. It is unsupervised, since the class label for each group is unknown and needs to be discovered. In data mining, it is utilised to understand the distribution of the data and inspect the distinctions between clusters. Moreover, it can be used as a preprocessing tool for other data mining methods, for example characterisation, attribute subset selection and classification.
Cluster analysis is used in various fields, including: biology, security, business intelligence, image pattern recognition, and Web search. It can be used to place customers into groups, organise projects into categories in project management and to sort Web search results into concise groups. Furthermore, it can be used to detect outliers, since these are located outside of clusters. The detection of outliers is useful in credit card fraud and for identifying criminal activity in e-commerce. 

According to \textcite{DataMiningAndPredictiveAnalytics}[524], clustering can also be used to prepare data (create clusters), for example for the input into neural networks.
\textcite{DataMiningAndPredictiveAnalytics}[524-525] explains, that data should be normalised before putting into a clustering algorithm, thus optimising the performance. Min-max normalization or Z-score standardization can be used to do so.
INSERT OTHER EXAMPLES HERE
%TODO: MORE ON OUTLIER DETECTION IN CHAPTER 12







%p 362 2nd to last paragraph - also example on using clustering in image recognition, image of numbers to partition each number into cluster

Clustering algorithms are used to create clusters, instead of humans. Consequently, groups of data can be unearthed, that were undiscovered before.

Distance measures are used to determine the similarities and dissimilarities between objects. 



