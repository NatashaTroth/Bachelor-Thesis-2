%https://books.google.at/books?hl=en&lr=&id=ZuIPv7OKm10C&oi=fnd&pg=PR5&dq=cluster+analysis&ots=7FXG9i4Yaa&sig=A0WB7hvq7hazTSWkYFrr_3iTAZU#v=onepage&q=cluster%20analysis&f=false



\textcite{hartigan1975clustering}[1] describes clustering as a means to group similar objects together. For example, two planets are considered similar, if (given measurement error) it is probable they could be perceived as the same planet. \textcite{romesburg2004cluster}[2] gives the gathering of a variety of pebbles and sorting them into piles of similar attributes (e.g. shape, size, colour) as an example of cluster analysis. \textcite{hartigan1975clustering}[1-3, 6] further explains, that it can be expected from similar objects, for them to act and be treated the same. Clustering is also used to name, display, summarise, predict, and require explanation of the objects in the cluster. If some of the objects assigned to a cluster exhibit certain properties, it is expected that the other objects in this cluster will exhibit them as well. Clustering is almost equivalent to classification. Real-world examples of clustering include classifications of animals, plants and diseases.

\textcite{han2011data}[361-363] state, that cluster analysis is another term for clustering. It divides a dataset of objects into subsets (clusters). The objects placed into one cluster are dissimilar to the objects assigned to other clusters. Therefore, such a cluster can also be defined as an implicit class. For this reason, clustering is occasionally referred to as automatic classification. The fact that cluster analysis can find groups by itself, gives it its unique advantage. As mentioned in section \ref{section:TheoryDataMining}, clustering is a type of unsupervised machine learning. It is unsupervised, since the class label for each group is unknown and needs to be discovered. In data mining, it is utilised to understand the distribution of the data and inspect the distinctions between clusters. Moreover, it can be used as a preprocessing tool for other data mining methods, such as characterisation, attribute subset selection, and classification.
Cluster analysis is used in various fields, including: biology, security, business intelligence, image pattern recognition, and web search. It can be used to place customers into groups, organise projects into categories in project management, and to sort web search results into concise groups. Furthermore, it can be used to detect outliers, since these are located outside of clusters (as pointed out in section \ref{section:NoisyData}).
%  The detection of outliers is useful in credit card fraud and for identifying criminal activity in e-commerce. 


% According to \textcite{hartigan1975clustering}[9-10], there are usually five different types of variables used in practice in clustering:
% \begin{itemize}
%   \item Counts: no arbitrary scale (e.g. number of legs on a spider) 
%   \item Ratio scale: only defined in proportion to a standard volume (e.g. volume of a liquid in a glass)
%   \item Interval scale: chosen from a standard position in a standard unit (e.g. height of a building)
%   \item Ordinal scale: ordered classification, can be changed by a monotonic transformation (e.g. socio-economic status)
%   \item Category scale: classification that can be adjusted by a one-to-one transformation (e.g. religion)
% \end{itemize}
% A dataset can be comprised of various variable types (\textit{mixed}), of the same type but with different ranges (\textit{heterogeneous}), or of variables with the same range (\textit{homogenous}). There are also methods for conducting type or scale conversions.

%data mining concepts..


% According to \textcite{DataMiningAndPredictiveAnalytics}[524], clustering can also be used to prepare data (create clusters), for example for the input into neural networks.
\textcite{DataMiningAndPredictiveAnalytics}[524-525] explain, that data should be normalised before being put into a clustering algorithm, thus optimising the performance. Min-max normalization or Z-score standardization can be used to do so (see section \ref{section:Normalisation}).
% TODO:INSERT OTHER EXAMPLES HERE
%TODO: MORE ON OUTLIER DETECTION IN CHAPTER 12







%p 362 2nd to last paragraph - also example on using clustering in image recognition, image of numbers to partition each number into cluster

%TODO: use these sentences?:
% Clustering algorithms are used to create clusters, instead of humans. Consequently, groups of data can be unearthed, that were undiscovered before.

% Distance measures are used to determine the similarities and dissimilarities between objects. 



