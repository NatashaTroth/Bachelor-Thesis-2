% in clustering after dim red

Dimensionality reduction was implemented to reduce the number of dimensions (number of attributes, so in this case number of columns). Principal Components Analysis (PCA) and t-SNE, as described in section \ref{section:DimensionalityReduction}, were used to reduce the dimensionality of the data set. PCA was the initial approach used in the experiment. The sklearn PCA\footnote{\url{https://scikit-learn.org/stable/modules/generated/sklearn.decomposition.PCA.html}} function was used to reduce the number of dimensions to 2, which simplified visualisation in 2D scatterplots. The PCA allowed 65\%-95\% (depending on data preparation type) of the data's important structures to be accounted for in only the first two or three principle components. As can be seen in figure \textbf{TODO: figures}, the resulting data from these components, didn't show any significant clusters, in comparison to the t-SNE results.
%&todo - this is because not linear data and pca is linear.
%todo - mention 3D or just stick to 2D

The t-SNE approach proved to be more significant.

Since the results 


\subsubsection{Find t-SNE parameters}
The first approach to find suitable t-SNE parameters to create the most distinct clusters possible, was to compare the results created using various parameter values. For this, several 2D t-SNE scatter plots were created using different values for the parameters perplexity, learning rate, and number of iterations. The t-SNE algorithm was implemented using the sklearn t-SNE\footnote{\url{https://scikit-learn.org/stable/modules/generated/sklearn.manifold.TSNE.html}} method. The number of iterations was originally set to the default value 1000, but was later raised to 5000, since the results were usually quite different. Raising the number of iterations provided more consistent results (the overall structure was the same, though usally rotated differently). 

Perplexity was the next parameter to select. \textcite{wattenberg2016how} proved to be a helpful source in tuning this parameter. Multiple values between the recommended 5 and 50 were tested. 

!!!TODO - RERUN PERPLEXITY WITH 40-50 TO BE SURE







To further confirm the t-SNE perplexity parameter, the equation reviewed in section \ref{section:tSNE} was implemented. However, the results were not as expected and so the results from the visual and evaluation score comparison were chosen.
