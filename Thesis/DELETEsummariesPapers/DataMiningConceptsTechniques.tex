

%BOOK Data mining concepts and techniques
%page 12 and 13 explains how much data exists, goes through google, basically why we need data mining
\textcite{han2011data}[16, 17] explain, that the term "data mining" is a misnomer. A more suitable phrase would be "knowledge mining from data". The word "mining" represents valuable nuggets found within large amounts of raw material. Other names used to describe the same process include: knowledge discovery from data (KDD), knowledge extraction, data/pattern analysis, data archaeology, and data dredging.
According to the authors, the discovery of data is an iterative process represented in the following steps

\begin{enumerate}
  \item Data cleaning
  \item Data integration (combine multiple data sources)
  \item Data selection (relevant data is extracted)
  \item Data transformation (into applicable forms for data mining )
  \item Data mining (discover patterns)
  \item Pattern evaluation (determine if patterns have a meaning)
  \item Knowledge presentation
\end{enumerate}

%page 18 - another sentence explaining what data mining is, but similar to one from other book
%page 18
Typical data forms used for mining can be database data, data warehouse data, and transactional data. Other forms include data streams, ordered/sequence data, graph or networked data, spatial data, text data, multimedia data, and the World Wide Web.

%page 28, 29 - outlier analysis
Outliers are objects that vary to the general behaviour or model of the data. In some cases, the uncommon events are of more interest. One of these instances is detecting unusually large payments compared to the card holders normal payments, to uncover fraudulent usage of credit cards.

%page 32 - there is an example for unsupervised m. l. on this page, but not really sure if need it
Using unsupervised machine learning, it is possible to detect classes within data.

