SmartEater is an upcoming mHealth (mobile health) smartphone app, with the goal to provide the user with content-dependent feedback, to avert a food craving episode. Using machine learning and  smartphone sensor and usage data, stress and upcoming cravings can be predicted. The goal of this thesis is to use unsupervised machine learning, more specifically cluster analysis, to find the ideal time delta to construct high quality clusterings from this data. The smartphone sensor and usage data recorded, includes accelerometer, volume of the audio, percentage of screen-on-time, number of notifications, light sensor values, and app usages in the categories “communication”, “video players”, and “other”. The data recorded from test subjects is aggregated into two datasets (1 hour and 3 hours), each with multiple time lengths. After cleaning and normalising these datasets, the number of dimensions is reduced from eight to two using t-SNE. Suitable t-SNE parameters are tuned to create the best possible t-SNE visualisation. The density-based clustering methods DBSCAN and OPTICS are used to group data points into clusters for each of the total ten time lengths. These results are compared using the mathematical evaluation scores Silhouette Score, Davies-Bouldin Index, and Caliński-Harabasz Index. The results indicate, that the time deltas 2 hours, 1 hour (both from the 3 hour dataset), 1 hour, and 30 minutes (both from the 1 hour dataset) create the most distinct and well-defined clusters.