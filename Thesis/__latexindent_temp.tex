% (150-300 Worte)
SmartEater ist eine bevorstehende mHealth-Smartphone-App (Mobile Health) mit dem Ziel, dem Benutzer inhaltsabhängiges Feedback zu geben, um eine Episode mit Verlangen nach Essen abzuwenden. Mithilfe von Machine Learning und Smartphone Sensor- und Nutzungsdaten können Stress und bevorstehendes Verlangen vorhergesagt werden. Das Ziel dieser Arbeit ist es, Unsupervised Machine Learning zu verwenden, um das beste Zeit Delta zu finden, um aus diesen Daten qualitativ hochwertige Cluster zu erstellen. Die aufgezeichneten Smartphone Sensor- und Nutzungsdaten umfassen den Beschleunigungsmesser, die Lautstärke des Audios, den Prozentsatz der Einschaltdauer des Bildschirms, die Anzahl der Benachrichtigungen, die Lichtsensorwerte und die App-Nutzung in den Kategorien „Kommunikation“, „Videoplayer“ und „Sonstige“. Die von Testpersonen aufgezeichneten Daten werden in zwei Datensätzen (1 Stunde und 3 Stunden) mit jeweils mehreren Zeitlängen aggregiert. Nach dem Bereinigen und Normalisieren dieser Datensätze, wird die Anzahl der Dimensionen, mithilfe von t-SNE, von acht auf zwei reduziert. Geeignete t-SNE-Parameter werden abgestimmt, um die bestmögliche t-SNE-Visualisierung zu erzielen. Die dichtebasierten Clustering-Methoden DBSCAN und OPTICS werden verwendet, um Datenpunkte für jede der insgesamt zehn Zeitlängen in Cluster zu gruppieren. Die Ergebnisse werden unter Verwendung der mathematischen Bewertungsmaße Silhouette Score, Davies-Bouldin Index und Caliński-Harabasz Index verglichen. Die Ergebnisse zeigen, dass die Zeit Deltas 30 Minuten (aus dem 1-Stunden-Datensatz), 1 Stunde und 2 Stunden (beide aus dem 3-Stunden-Datensatz) die deutlichsten und klarsten Cluster bilden.