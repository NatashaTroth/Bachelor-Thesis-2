\begin{appendices}
\textbf{\color{red} Anhänge löschen, die nicht verwendet werden.}

\section{t-SNE parameters comparison figures}
\label{appendix:tSNEParameters}
\subsection{Perplexity}
\label{appendix:tSNEParametersPerplexity}
% 
In the following figures, the t-SNE results of different perplexitites are compared, for the different time length files (1h and 3h), using the first columns of each feature (1h: first 15 minutes, 3h: fist 30 minutes ). The left scatter plots depict t-SNE results, the right scatter plots visualise DBSCAN clusterings of t-SNE results).

\subsubsection{Perplexity = 5}
%------------------ PERPLEXITY 10: ------------------
% -- 1h, perp 5 --
\begin{figure}[H]
  \centering
  \begin{subfigure}{.5\textwidth}
    \centering
    \includegraphics[width=0.9\textwidth]{./images/tsneParametersTest/perplexity/perp5-1hTSNE.png}
  % \caption{}
  % \label{figure:}
  \end{subfigure}%
  \begin{subfigure}{.5\textwidth}
    \centering
    \includegraphics[width=0.9\textwidth]{./images/tsneParametersTest/perplexity/perp5-1hDBSCAN.png}
    % \caption{}
    % \label{figure:}
  \end{subfigure}
	\caption{\textbf{1h} data files, t-SNE calculated with the following parameters: \textbf{perplexity=5}, n\_iter=5000, learning\_rate=50}
	\label{figure:1hperp5TSNE}
\end{figure}


% -- 3h, perp 5 --
\begin{figure}[H]
	\centering
	
  \centering
	\begin{subfigure}{.5\textwidth}
    \centering
    \includegraphics[width=0.9\textwidth]{./images/tsneParametersTest/perplexity/perp5-3hTSNE.png}
  % \caption{}
  % \label{figure:}
  \end{subfigure}%
  \begin{subfigure}{.5\textwidth}
    \centering
    \includegraphics[width=0.9\textwidth]{./images/tsneParametersTest/perplexity/perp5-3hDBSCAN.png}
    % \caption{}
    % \label{figure:}
	\end{subfigure}
	\caption{\textbf{3h} data files, t-SNE calculated with the following parameters: \textbf{perplexity=5}, n\_iter=5000, learning\_rate=50}
  \label{figure:3hperp5TSNE}
\end{figure}


%------------------ PERPLEXITY 10: ------------------
\subsubsection{Perplexity = 10}
% -- 1h, perp 10 --
\begin{figure}[H]
  \centering
  \begin{subfigure}{.5\textwidth}
    \centering
    \includegraphics[width=0.9\textwidth]{./images/tsneParametersTest/perplexity/perp10-1hTSNE.png}
  % \caption{}
  % \label{figure:}
  \end{subfigure}%
  \begin{subfigure}{.5\textwidth}
    \centering
    \includegraphics[width=0.9\textwidth]{./images/tsneParametersTest/perplexity/perp10-1hDBSCAN.png}
    % \caption{}
    % \label{figure:}
  \end{subfigure}
	\caption{\textbf{1h} data files, t-SNE calculated with the following parameters: \textbf{perplexity=10}, n\_iter=5000, learning\_rate=50}
  \label{figure:1hperp10TSNE}
\end{figure}

% -- 3h, perp 10 --
\begin{figure}[H]
  \centering
	\begin{subfigure}{.5\textwidth}
    \centering
    \includegraphics[width=0.9\textwidth]{./images/tsneParametersTest/perplexity/perp10-3hTSNE.png}
  % \caption{}
  % \label{figure:}
  \end{subfigure}%
  \begin{subfigure}{.5\textwidth}
    \centering
    \includegraphics[width=0.9\textwidth]{./images/tsneParametersTest/perplexity/perp10-3hDBSCAN.png}
    % \caption{}
    % \label{figure:}
	\end{subfigure}
	\caption{\textbf{3h} data files, t-SNE calculated with the following parameters: \textbf{perplexity=10}, n\_iter=5000, learning\_rate=50}
  \label{figure:3hperp10TSNE}
\end{figure}

%------------------ PERPLEXITY 20: ------------------
\subsubsection{Perplexity = 20}
% -- 1h, perp 20 --
\begin{figure}[H]
  \centering
  \begin{subfigure}{.5\textwidth}
    \centering
    \includegraphics[width=0.9\textwidth]{./images/tsneParametersTest/perplexity/perp20-1hTSNE.png}
  % \caption{}
  % \label{figure:}
  \end{subfigure}%
  \begin{subfigure}{.5\textwidth}
    \centering
    \includegraphics[width=0.9\textwidth]{./images/tsneParametersTest/perplexity/perp20-1hDBSCAN.png}
    % \caption{}
    % \label{figure:}
  \end{subfigure}
	\caption{\textbf{1h} data files, t-SNE calculated with the following parameters: \textbf{perplexity=20}, n\_iter=5000, learning\_rate=50}
  \label{figure:1hperp20TSNE}
\end{figure}

% -- 3h, perp 20 --
\begin{figure}[H]
  \centering
	\begin{subfigure}{.5\textwidth}
    \centering
    \includegraphics[width=0.9\textwidth]{./images/tsneParametersTest/perplexity/perp20-3hTSNE.png}
  % \caption{}
  % \label{figure:}
  \end{subfigure}%
  \begin{subfigure}{.5\textwidth}
    \centering
    \includegraphics[width=0.9\textwidth]{./images/tsneParametersTest/perplexity/perp20-3hDBSCAN.png}
    % \caption{}
    % \label{figure:}
	\end{subfigure}
	\caption{\textbf{3h} data files, t-SNE calculated with the following parameters: \textbf{perplexity=20}, n\_iter=5000, learning\_rate=50}
  \label{figure:3hperp20TSNE}
\end{figure}



%------------------ PERPLEXITY 30: ------------------
\subsubsection{Perplexity = 30}
% -- 1h, perp 30 --
\begin{figure}[H]
  \centering
  \begin{subfigure}{.5\textwidth}
    \centering
    \includegraphics[width=0.9\textwidth]{./images/tsneParametersTest/perplexity/perp30-1hTSNE.png}
  % \caption{}
  % \label{figure:}
  \end{subfigure}%
  \begin{subfigure}{.5\textwidth}
    \centering
    \includegraphics[width=0.9\textwidth]{./images/tsneParametersTest/perplexity/perp30-1hDBSCAN.png}
    % \caption{}
    % \label{figure:}
  \end{subfigure}
	\caption{\textbf{1h} data files, t-SNE calculated with the following parameters: \textbf{perplexity=30}, n\_iter=5000, learning\_rate=50}
  \label{figure:1hperp30TSNE}
\end{figure}

% -- 3h, perp 30 --
\begin{figure}[H]
  \centering
	\begin{subfigure}{.5\textwidth}
    \centering
    \includegraphics[width=0.9\textwidth]{./images/tsneParametersTest/perplexity/perp30-3hTSNE.png}
  % \caption{}
  % \label{figure:}
  \end{subfigure}%
  \begin{subfigure}{.5\textwidth}
    \centering
    \includegraphics[width=0.9\textwidth]{./images/tsneParametersTest/perplexity/perp30-3hDBSCAN.png}
    % \caption{}
    % \label{figure:}
	\end{subfigure}
	\caption{\textbf{3h} data files, t-SNE calculated with the following parameters: \textbf{perplexity=30}, n\_iter=5000, learning\_rate=50}
  \label{figure:3hperp30TSNE}
\end{figure}



%------------------ PERPLEXITY 40: ------------------
\subsubsection{Perplexity = 40}
% -- 1h, perp 40 --
\begin{figure}[H]
  \centering
  \begin{subfigure}{.5\textwidth}
    \centering
    \includegraphics[width=0.9\textwidth]{./images/tsneParametersTest/perplexity/perp40-1hTSNE.png}
  % \caption{}
  % \label{figure:}
  \end{subfigure}%
  \begin{subfigure}{.5\textwidth}
    \centering
    \includegraphics[width=0.9\textwidth]{./images/tsneParametersTest/perplexity/perp40-1hDBSCAN.png}
    % \caption{}
    % \label{figure:}
  \end{subfigure}
	\caption{\textbf{1h} data files, t-SNE calculated with the following parameters: \textbf{perplexity=40}, n\_iter=5000, learning\_rate=50}
  \label{figure:1hperp40TSNE}
\end{figure}

% -- 3h, perp 40 --
\begin{figure}[H]
  \centering
	\begin{subfigure}{.5\textwidth}
    \centering
    \includegraphics[width=0.9\textwidth]{./images/tsneParametersTest/perplexity/perp40-3hTSNE.png}
  % \caption{}
  % \label{figure:}
  \end{subfigure}%
  \begin{subfigure}{.5\textwidth}
    \centering
    \includegraphics[width=0.9\textwidth]{./images/tsneParametersTest/perplexity/perp40-3hDBSCAN.png}
    % \caption{}
    % \label{figure:}
	\end{subfigure}
	\caption{\textbf{3h} data files, t-SNE calculated with the following parameters: \textbf{perplexity=40}, n\_iter=5000, learning\_rate=50}
  \label{figure:3hperp40TSNE}
\end{figure}



%------------------ PERPLEXITY 45: ------------------
\subsubsection{Perplexity = 45}
% -- 1h, perp 45 --
\begin{figure}[H]
  \centering
  \begin{subfigure}{.5\textwidth}
    \centering
    \includegraphics[width=0.9\textwidth]{./images/tsneParametersTest/perplexity/perp45-1hTSNE.png}
  % \caption{}
  % \label{figure:}
  \end{subfigure}%
  \begin{subfigure}{.5\textwidth}
    \centering
    \includegraphics[width=0.9\textwidth]{./images/tsneParametersTest/perplexity/perp45-1hDBSCAN.png}
    % \caption{}
    % \label{figure:}
  \end{subfigure}
	\caption{\textbf{1h} data files, t-SNE calculated with the following parameters: \textbf{perplexity=45}, n\_iter=5000, learning\_rate=50}
  \label{figure:1hperp45TSNE}
\end{figure}

% -- 3h, perp 45 --
\begin{figure}[H]
  \centering
	\begin{subfigure}{.5\textwidth}
    \centering
    \includegraphics[width=0.9\textwidth]{./images/tsneParametersTest/perplexity/perp45-3hTSNE.png}
  % \caption{}
  % \label{figure:}
  \end{subfigure}%
  \begin{subfigure}{.5\textwidth}
    \centering
    \includegraphics[width=0.9\textwidth]{./images/tsneParametersTest/perplexity/perp45-3hDBSCAN.png}
    % \caption{}
    % \label{figure:}
	\end{subfigure}
	\caption{\textbf{3h} data files, t-SNE calculated with the following parameters: \textbf{perplexity=45}, n\_iter=5000, learning\_rate=50}
  \label{figure:3hperp45TSNE}
\end{figure}



%------------------ PERPLEXITY 50: ------------------
\subsubsection{Perplexity = 50}
% -- 1h, perp 50 --
\begin{figure}[H]
  \centering
  \begin{subfigure}{.5\textwidth}
    \centering
    \includegraphics[width=0.9\textwidth]{./images/tsneParametersTest/perplexity/perp50-1hTSNE.png}
  % \caption{}
  % \label{figure:}
  \end{subfigure}%
  \begin{subfigure}{.5\textwidth}
    \centering
    \includegraphics[width=0.9\textwidth]{./images/tsneParametersTest/perplexity/perp50-1hDBSCAN.png}
    % \caption{}
    % \label{figure:}
  \end{subfigure}
	\caption{\textbf{1h} data files, t-SNE calculated with the following parameters: \textbf{perplexity=50}, n\_iter=5000, learning\_rate=50}
  \label{figure:1hperp50TSNE}
\end{figure}

% -- 3h, perp 50 --
\begin{figure}[H]
  \centering
	\begin{subfigure}{.5\textwidth}
    \centering
    \includegraphics[width=0.9\textwidth]{./images/tsneParametersTest/perplexity/perp50-3hTSNE.png}
  % \caption{}
  % \label{figure:}
  \end{subfigure}%
  \begin{subfigure}{.5\textwidth}
    \centering
    \includegraphics[width=0.9\textwidth]{./images/tsneParametersTest/perplexity/perp50-3hDBSCAN.png}
    % \caption{}
    % \label{figure:}
	\end{subfigure}
	\caption{\textbf{3h} data files, t-SNE calculated with the following parameters: \textbf{perplexity=50}, n\_iter=5000, learning\_rate=50}
  \label{figure:3hperp50TSNE}
\end{figure}





\subsubsection{Perplexity Comparison Results (Average of two different t-SNE runs)}
\label{appendix:compareAveragePerplexity}

\begin{figure}[H]
  \centering
  \includegraphics[width=0.8\textwidth]{./images/tsneParametersTest/perplexity/perplexityEvaluationScoresAverage.png}
  \caption{Comparison of Silhouette Coefficient, Davies-Bouldin Index, and Caliński-Harabasz Index for different t-SNE \textbf{perplexities}. The lighter green highlighted values indicate the best values of that file aggregation (1h or 3h files). The dark green highlighted values illustrate the overall best values over all files (1h and 3h files).}
  \label{figure:perplexityEvaluationScoresAverage}
\end{figure}

\begin{figure}[H]
  \centering
  \includegraphics[width=0.4\textwidth]{./images/tsneParametersTest/perplexity/perplexityEvaluationScoresDetailedAverage.png}
  \caption{Comparison of Silhouette Coefficient, Davies-Bouldin Index, and Caliński-Harabasz Index for different t-SNE \textbf{perplexities}. The lighter green highlighted values indicate the best values of that file aggregation (1h or 3h files). The dark green highlighted values illustrate the overall best values over all files (1h and 3h files).}
  \label{figure:perplexityEvaluationScoresDetailedAverage}
\end{figure}


\clearpage

\subsection{Learning Rate}
\label{appendix:tSNEParametersLearningRate}
% 
% In the following figures, the t-SNE results of different perplexitites are compared, for the different time length files (1h and 3h), using the first columns of each feature (1h: first 15 minutes, 3h: fist 30 minutes ). The left scatter plots depict t-SNE results, the right scatter plots visualise DBSCAN clusterings of t-SNE results).

%------------------ LEARNING RATE 10: ------------------
\subsubsection{Learning Rate = 10}
% -- 1h, lr 10 --
\begin{figure}[H]
  \centering
  \begin{subfigure}{.5\textwidth}
    \centering
    \includegraphics[width=0.9\textwidth]{./images/tsneParametersTest/learningRate/lr10-1hTSNE.png}
  % \caption{}
  % \label{figure:}
  \end{subfigure}%
  \begin{subfigure}{.5\textwidth}
    \centering
    \includegraphics[width=0.9\textwidth]{./images/tsneParametersTest/learningRate/lr10-1hDBSCAN.png}
    % \caption{}
    % \label{figure:}
  \end{subfigure}
	\caption{\textbf{1h} data files, t-SNE calculated with the following parameters: perplexity=40, n\_iter=5000, \textbf{learning\_rate=10}}
	\label{figure:1hlr10TSNE}
\end{figure}

% -- 3h, lr 10 --
\begin{figure}[H]
	\centering
	
  \centering
	\begin{subfigure}{.5\textwidth}
    \centering
    \includegraphics[width=0.9\textwidth]{./images/tsneParametersTest/learningRate/lr10-3hTSNE.png}
  % \caption{}
  % \label{figure:}
  \end{subfigure}%
  \begin{subfigure}{.5\textwidth}
    \centering
    \includegraphics[width=0.9\textwidth]{./images/tsneParametersTest/learningRate/lr10-3hDBSCAN.png}
    % \caption{}
    % \label{figure:}
	\end{subfigure}
	\caption{\textbf{3h} data files, t-SNE calculated with the following parameters: perplexity=40, n\_iter=5000, \textbf{learning\_rate=10}}
  \label{figure:3hlr10TSNE}
\end{figure}

%------------------ LEARNING RATE 200: ------------------
\subsubsection{Learning Rate = 200}
% -- 1h, lr 200 --
\begin{figure}[H]
  \centering
  \begin{subfigure}{.5\textwidth}
    \centering
    \includegraphics[width=0.9\textwidth]{./images/tsneParametersTest/learningRate/lr200-1hTSNE.png}
  % \caption{}
  % \label{figure:}
  \end{subfigure}%
  \begin{subfigure}{.5\textwidth}
    \centering
    \includegraphics[width=0.9\textwidth]{./images/tsneParametersTest/learningRate/lr200-1hDBSCAN.png}
    % \caption{}
    % \label{figure:}
  \end{subfigure}
	\caption{\textbf{1h} data files, t-SNE calculated with the following parameters: perplexity=40, n\_iter=5000, \textbf{learning\_rate=200}}
	\label{figure:1hlr200TSNE}
\end{figure}

% -- 3h, lr 200 --
\begin{figure}[H]
	\centering
	
  \centering
	\begin{subfigure}{.5\textwidth}
    \centering
    \includegraphics[width=0.9\textwidth]{./images/tsneParametersTest/learningRate/lr200-3hTSNE.png}
  % \caption{}
  % \label{figure:}
  \end{subfigure}%
  \begin{subfigure}{.5\textwidth}
    \centering
    \includegraphics[width=0.9\textwidth]{./images/tsneParametersTest/learningRate/lr200-3hDBSCAN.png}
    % \caption{}
    % \label{figure:}
	\end{subfigure}
	\caption{\textbf{3h} data files, t-SNE calculated with the following parameters: perplexity=40, n\_iter=5000, \textbf{learning\_rate=200}}
  \label{figure:3hlr200TSNE}
\end{figure}


%------------------ LEARNING RATE 400: ------------------
\subsubsection{Learning Rate = 400}
% -- 1h, lr 400 --
\begin{figure}[H]
  \centering
  \begin{subfigure}{.5\textwidth}
    \centering
    \includegraphics[width=0.9\textwidth]{./images/tsneParametersTest/learningRate/lr400-1hTSNE.png}
  % \caption{}
  % \label{figure:}
  \end{subfigure}%
  \begin{subfigure}{.5\textwidth}
    \centering
    \includegraphics[width=0.9\textwidth]{./images/tsneParametersTest/learningRate/lr400-1hDBSCAN.png}
    % \caption{}
    % \label{figure:}
  \end{subfigure}
	\caption{\textbf{1h} data files, t-SNE calculated with the following parameters: perplexity=40, n\_iter=5000, \textbf{learning\_rate=400}}
	\label{figure:1hlr400TSNE}
\end{figure}

% -- 3h, lr 400 --
\begin{figure}[H]
	\centering
	
  \centering
	\begin{subfigure}{.5\textwidth}
    \centering
    \includegraphics[width=0.9\textwidth]{./images/tsneParametersTest/learningRate/lr400-3hTSNE.png}
  % \caption{}
  % \label{figure:}
  \end{subfigure}%
  \begin{subfigure}{.5\textwidth}
    \centering
    \includegraphics[width=0.9\textwidth]{./images/tsneParametersTest/learningRate/lr400-3hDBSCAN.png}
    % \caption{}
    % \label{figure:}
	\end{subfigure}
	\caption{\textbf{3h} data files, t-SNE calculated with the following parameters: perplexity=40, n\_iter=5000, \textbf{learning\_rate=400}}
  \label{figure:3hlr400TSNE}
\end{figure}


%------------------ LEARNING RATE 600: ------------------
\subsubsection{Learning Rate = 600}
% -- 1h, lr 600 --
\begin{figure}[H]
  \centering
  \begin{subfigure}{.5\textwidth}
    \centering
    \includegraphics[width=0.9\textwidth]{./images/tsneParametersTest/learningRate/lr600-1hTSNE.png}
  % \caption{}
  % \label{figure:}
  \end{subfigure}%
  \begin{subfigure}{.5\textwidth}
    \centering
    \includegraphics[width=0.9\textwidth]{./images/tsneParametersTest/learningRate/lr600-1hDBSCAN.png}
    % \caption{}
    % \label{figure:}
  \end{subfigure}
	\caption{\textbf{1h} data files, t-SNE calculated with the following parameters: perplexity=40, n\_iter=5000, \textbf{learning\_rate=600}}
	\label{figure:1hlr600TSNE}
\end{figure}

% -- 3h, lr 600 --
\begin{figure}[H]
	\centering
	
  \centering
	\begin{subfigure}{.5\textwidth}
    \centering
    \includegraphics[width=0.9\textwidth]{./images/tsneParametersTest/learningRate/lr600-3hTSNE.png}
  % \caption{}
  % \label{figure:}
  \end{subfigure}%
  \begin{subfigure}{.5\textwidth}
    \centering
    \includegraphics[width=0.9\textwidth]{./images/tsneParametersTest/learningRate/lr600-3hDBSCAN.png}
    % \caption{}
    % \label{figure:}
	\end{subfigure}
	\caption{\textbf{3h} data files, t-SNE calculated with the following parameters: perplexity=40, n\_iter=5000, \textbf{learning\_rate=600}}
  \label{figure:3hlr600TSNE}
\end{figure}




%------------------ LEARNING RATE 800: ------------------
\subsubsection{Learning Rate = 800}
% -- 1h, lr 800 --
\begin{figure}[H]
  \centering
  \begin{subfigure}{.5\textwidth}
    \centering
    \includegraphics[width=0.9\textwidth]{./images/tsneParametersTest/learningRate/lr800-1hTSNE.png}
  % \caption{}
  % \label{figure:}
  \end{subfigure}%
  \begin{subfigure}{.5\textwidth}
    \centering
    \includegraphics[width=0.9\textwidth]{./images/tsneParametersTest/learningRate/lr800-1hDBSCAN.png}
    % \caption{}
    % \label{figure:}
  \end{subfigure}
	\caption{\textbf{1h} data files, t-SNE calculated with the following parameters: perplexity=40, n\_iter=5000, \textbf{learning\_rate=800}}
	\label{figure:1hlr800TSNE}
\end{figure}

% -- 3h, lr 800 --
\begin{figure}[H]
	\centering
	
  \centering
	\begin{subfigure}{.5\textwidth}
    \centering
    \includegraphics[width=0.9\textwidth]{./images/tsneParametersTest/learningRate/lr800-3hTSNE.png}
  % \caption{}
  % \label{figure:}
  \end{subfigure}%
  \begin{subfigure}{.5\textwidth}
    \centering
    \includegraphics[width=0.9\textwidth]{./images/tsneParametersTest/learningRate/lr800-3hDBSCAN.png}
    % \caption{}
    % \label{figure:}
	\end{subfigure}
	\caption{\textbf{3h} data files, t-SNE calculated with the following parameters: perplexity=40, n\_iter=5000, \textbf{learning\_rate=800}}
  \label{figure:3hlr800TSNE}
\end{figure}


%------------------ LEARNING RATE 1000: ------------------
\subsubsection{Learning Rate = 1000}
% -- 1h, lr 1000 --
\begin{figure}[H]
  \centering
  \begin{subfigure}{.5\textwidth}
    \centering
    \includegraphics[width=0.9\textwidth]{./images/tsneParametersTest/learningRate/lr1000-1hTSNE.png}
  % \caption{}
  % \label{figure:}
  \end{subfigure}%
  \begin{subfigure}{.5\textwidth}
    \centering
    \includegraphics[width=0.9\textwidth]{./images/tsneParametersTest/learningRate/lr1000-1hDBSCAN.png}
    % \caption{}
    % \label{figure:}
  \end{subfigure}
	\caption{\textbf{1h} data files, t-SNE calculated with the following parameters: perplexity=40, n\_iter=5000, \textbf{learning\_rate=1000}}
	\label{figure:1hlr1000TSNE}
\end{figure}

% -- 3h, lr 1000 --
\begin{figure}[H]
	\centering
	
  \centering
	\begin{subfigure}{.5\textwidth}
    \centering
    \includegraphics[width=0.9\textwidth]{./images/tsneParametersTest/learningRate/lr1000-3hTSNE.png}
  % \caption{}
  % \label{figure:}
  \end{subfigure}%
  \begin{subfigure}{.5\textwidth}
    \centering
    \includegraphics[width=0.9\textwidth]{./images/tsneParametersTest/learningRate/lr1000-3hDBSCAN.png}
    % \caption{}
    % \label{figure:}
	\end{subfigure}
	\caption{\textbf{3h} data files, t-SNE calculated with the following parameters: perplexity=40, n\_iter=5000, \textbf{learning\_rate=1000}}
  \label{figure:3hlr1000TSNE}
\end{figure}



\subsubsection{Learning Rate Detailed Comparison Results }
\label{appendix:comparelearningRateDetailed}

\begin{figure}
  \centering
  \includegraphics[width=0.8\textwidth]{./images/tsneParametersTest/learningRate/learningRateEvaluationScoresDetailed.png}
  \caption{Comparison of Silhouette Coefficient, Davies-Bouldin Index, and Caliński-Harabasz Index for different t-SNE \textbf{learning rate} values.}
  \label{figure:learningRateEvaluationScoresDetailed}
\end{figure}

\begin{figure}
  \centering
  \includegraphics[width=0.8\textwidth]{./images/tsneParametersTest/learningRate/learningRateEvaluationScoresDetailed2.png}
  \caption{Comparison of Silhouette Coefficient, Davies-Bouldin Index, and Caliński-Harabasz Index for different t-SNE \textbf{learning rate} values.}
  \label{figure:learningRateEvaluationScoresDetailed2}
\end{figure}

%.................................COMPARISON AVERAGES..................................
\subsubsection{Learning Rate Comparison Results (Average of two different t-SNE runs)}
\label{appendix:compareAverageLearningRate}


\begin{figure}[H]
  \centering
  \includegraphics[width=0.8\textwidth]{./images/tsneParametersTest/learningRate/learningRateEvaluationScoresAverage.png}
  \caption{Comparison of Silhouette Coefficient, Davies-Bouldin Index, and Caliński-Harabasz Index for different t-SNE \textbf{learning rate} values.}
  \label{figure:learningRateEvaluationScoresAverage}
\end{figure}

\begin{figure}[H]
  \centering
  \includegraphics[width=0.8\textwidth]{./images/tsneParametersTest/learningRate/learningRateEvaluationScoresAverageDetailed.png}
  \caption{Comparison of Silhouette Coefficient, Davies-Bouldin Index, and Caliński-Harabasz Index for different t-SNE \textbf{learning rate} values.}
  \label{figure:learningRateEvaluationScoresAverageDetailed}
\end{figure}

\begin{figure}[H]
  \centering
  \includegraphics[width=0.8\textwidth]{./images/tsneParametersTest/learningRate/learningRateEvaluationScoresAverageDetailed2.png}
  \caption{Comparison of Silhouette Coefficient, Davies-Bouldin Index, and Caliński-Harabasz Index for different t-SNE \textbf{learning rate} values.}
  \label{figure:learningRateEvaluationScoresAverageDetailed2}
\end{figure}

\begin{figure}[H]
  \centering
  \includegraphics[width=0.8\textwidth]{./images/tsneParametersTest/learningRate/learningRateEvaluationScoresAverageDetailed3.png}
  \caption{Comparison of Silhouette Coefficient, Davies-Bouldin Index, and Caliński-Harabasz Index for different t-SNE \textbf{learning rate} values.}
  \label{figure:learningRateEvaluationScoresAverageDetailed3}
\end{figure}






\subsubsection{Learning Rate Comparison of 20 and 800}
\label{appendig:compareLearningRate20and800}

\begin{figure}[H]
  \centering
  \includegraphics[width=1\textwidth]{./images/tsneParametersTest/learningRate/learningRateEvaluationScoresAverageDetailed4.png}
  \caption{Comparison of Silhouette Coefficient, Davies-Bouldin Index, and Caliński-Harabasz Index for the t-SNE \textbf{learning rate} values 20 and 80.}
  \label{figure:learningRateEvaluationScoresAverageDetailed4}
\end{figure}

%------------------ 1h: ------------------
\begin{figure}[H]
  \centering
  \begin{subfigure}{.5\textwidth}
    \centering
    \includegraphics[width=0.9\textwidth]{./images/tsneParametersTest/learningRate/lr201h-DBSCANCompare.png}
  % \caption{}
  % \label{figure:}
  \end{subfigure}%
  \begin{subfigure}{.5\textwidth}
    \centering
    \includegraphics[width=0.9\textwidth]{./images/tsneParametersTest/learningRate/lr8001h-DBSCANCompare.png}
    % \caption{}
    % \label{figure:}
  \end{subfigure}
	\caption{\textbf{1h} data files comparison of learning rate: a) 20, b) 800}
	\label{figure:1h-learningRateComparison20and800}
\end{figure}
%------------------ 3h: ------------------
\begin{figure}[H]
  \centering
  \begin{subfigure}{.5\textwidth}
    \centering
    \includegraphics[width=0.9\textwidth]{./images/tsneParametersTest/learningRate/lr203h-DBSCANCompare.png}
  % \caption{}
  % \label{figure:}
  \end{subfigure}%
  \begin{subfigure}{.5\textwidth}
    \centering
    \includegraphics[width=0.9\textwidth]{./images/tsneParametersTest/learningRate/lr8003h-DBSCANCompare.png}
    % \caption{}
    % \label{figure:}
  \end{subfigure}
	\caption{\textbf{3h} data files comparison of learning rate: a) 20, b) 800}
	\label{figure:3h-learningRateComparison20and800}
\end{figure}


\clearpage


\subsection{Clustering results}
\label{appendix:clusteringResults}
% In the following scatter plots, data points coloured black are indicated as noise.

\subsubsection{1h aggregated data files}

\begin{figure}[H]
	\centering
	\begin{subfigure}{.5\textwidth}
    \centering
    \includegraphics[width=0.9\textwidth]{./images/clusteringResults/1h-1-DBSCAN.png}
  \end{subfigure}%
  \begin{subfigure}{.5\textwidth}
    \centering
    \includegraphics[width=0.9\textwidth]{./images/clusteringResults/1h-1-OPTICS.png}
	\end{subfigure}
	\caption{Comparison of the scatter plots from the DBSCAN (a) and OPTICS (b) clusterings of the 1st column, so the first \textbf{15 minutes} (1h data files: first 15 minutes).}
  \label{figure:finalClustering1h-1}
\end{figure}

\begin{figure}[H]
	\centering
	\begin{subfigure}{.5\textwidth}
    \centering
    \includegraphics[width=0.9\textwidth]{./images/clusteringResults/1h-2-DBSCAN.png}
  \end{subfigure}%
  \begin{subfigure}{.5\textwidth}
    \centering
    \includegraphics[width=0.9\textwidth]{./images/clusteringResults/1h-2-OPTICS.png}
	\end{subfigure}
	\caption{Comparison of the scatter plots from the DBSCAN (a) and OPTICS (b) clusterings of the average of the 1st column and 2nd column, so the first \textbf{30 minutes} (1h data files: 15 minutes \& 30 minutes).}
  \label{figure:finalClustering1h-2}
\end{figure}

\begin{figure}[H]
	\centering
	\begin{subfigure}{.5\textwidth}
    \centering
    \includegraphics[width=0.9\textwidth]{./images/clusteringResults/1h-3-DBSCAN.png}
  \end{subfigure}%
  \begin{subfigure}{.5\textwidth}
    \centering
    \includegraphics[width=0.9\textwidth]{./images/clusteringResults/1h-3-OPTICS.png}
	\end{subfigure}
	\caption{Comparison of the scatter plots from the DBSCAN (a) and OPTICS (b) clusterings of the average of the 1st column to the 3rd column, so the first \textbf{45 minutes} (1h data files: 15 minutes, 30 minutes \& 45 minutes).}
  \label{figure:finalClustering1h-3}
\end{figure}

\begin{figure}[H]
	\centering
	\begin{subfigure}{.5\textwidth}
    \centering
    \includegraphics[width=0.9\textwidth]{./images/clusteringResults/1h-4-DBSCAN.png}
  \end{subfigure}%
  \begin{subfigure}{.5\textwidth}
    \centering
    \includegraphics[width=0.9\textwidth]{./images/clusteringResults/1h-4-OPTICS.png}
	\end{subfigure}
	\caption{Comparison of the scatter plots from the DBSCAN (a) and OPTICS (b) clusterings of the average of the 1st column to the 4th column, so the whole \textbf{1 hour} (1h data files: 15 minutes, 30 minutes, 45 minutes \& 1 hour).}
  \label{figure:finalClustering1h-4}
\end{figure}









\subsubsection{3h aggregated data files}

\begin{figure}[H]
	\centering
	\begin{subfigure}{.5\textwidth}
    \centering
    \includegraphics[width=0.9\textwidth]{./images/clusteringResults/3h-1-DBSCAN.png}
  \end{subfigure}%
  \begin{subfigure}{.5\textwidth}
    \centering
    \includegraphics[width=0.9\textwidth]{./images/clusteringResults/3h-1-OPTICS.png}
	\end{subfigure}
	\caption{Comparison of the scatter plots from the DBSCAN (a) and OPTICS (b) clusterings of the 1st column, so the first \textbf{30 minutes} (3h data files: first 30 minutes).}
  \label{figure:finalClustering3h-1}
\end{figure}



\begin{figure}[H]
	\centering
	\begin{subfigure}{.5\textwidth}
    \centering
    \includegraphics[width=0.9\textwidth]{./images/clusteringResults/3h-2-DBSCAN.png}
  \end{subfigure}%
  \begin{subfigure}{.5\textwidth}
    \centering
    \includegraphics[width=0.9\textwidth]{./images/clusteringResults/3h-2-OPTICS.png}
	\end{subfigure}
	\caption{Comparison of the scatter plots from the DBSCAN (a) and OPTICS (b) clusterings of the average of the 1st column and 2nd column, so the first \textbf{1 hour} (3h data files: 30 minutes \& 1 hour).}
  \label{figure:finalClustering3h-2}
\end{figure}

\begin{figure}[H]
	\centering
	\begin{subfigure}{.5\textwidth}
    \centering
    \includegraphics[width=0.9\textwidth]{./images/clusteringResults/3h-3-DBSCAN.png}
  \end{subfigure}%
  \begin{subfigure}{.5\textwidth}
    \centering
    \includegraphics[width=0.9\textwidth]{./images/clusteringResults/3h-3-OPTICS.png}
	\end{subfigure}
	\caption{Comparison of the scatter plots from the DBSCAN (a) and OPTICS (b) clusterings of the average of the 1st column to the 3rd column, so the first \textbf{1.5 hours} (3h data files: 30 minutes, 1 hour \& 1 hour 30 minutes).}
  \label{figure:finalClustering3h-3}
\end{figure}

\begin{figure}[H]
	\centering
	\begin{subfigure}{.5\textwidth}
    \centering
    \includegraphics[width=0.9\textwidth]{./images/clusteringResults/3h-4-DBSCAN.png}
  \end{subfigure}%
  \begin{subfigure}{.5\textwidth}
    \centering
    \includegraphics[width=0.9\textwidth]{./images/clusteringResults/3h-4-OPTICS.png}
	\end{subfigure}
	\caption{Comparison of the scatter plots from the DBSCAN (a) and OPTICS (b) clusterings of the average of the 1st column to the 4th column, so the first \textbf{2 hours} (3h data files: 30 minutes, 1 hour, 1 hour 30 minutes \& 2 hours).}
  \label{figure:finalClustering3h-4}
\end{figure}


\begin{figure}[H]
	\centering
	\begin{subfigure}{.5\textwidth}
    \centering
    \includegraphics[width=0.9\textwidth]{./images/clusteringResults/3h-5-DBSCAN.png}
  \end{subfigure}%
  \begin{subfigure}{.5\textwidth}
    \centering
    \includegraphics[width=0.9\textwidth]{./images/clusteringResults/3h-5-OPTICS.png}
	\end{subfigure}
	\caption{Comparison of the scatter plots from the DBSCAN (a) and OPTICS (b) clusterings of the average of the 1st column to the 5th column, so the first \textbf{2.5 hours} (3h data files: 30 minutes, 1 hour, 1 hour 30 minutes, 2 hours \& 2 hours 30 minutes).}
  \label{figure:finalClustering3h-5}
\end{figure}


\begin{figure}[H]
	\centering
	\begin{subfigure}{.5\textwidth}
    \centering
    \includegraphics[width=0.9\textwidth]{./images/clusteringResults/3h-6-DBSCAN.png}
  \end{subfigure}%
  \begin{subfigure}{.5\textwidth}
    \centering
    \includegraphics[width=0.9\textwidth]{./images/clusteringResults/3h-6-OPTICS.png}
	\end{subfigure}
	\caption{Comparison of the scatter plots from the DBSCAN (a) and OPTICS (b) clusterings of the average of the 1st column to the 6th column, so all \textbf{3 hours} (3h data files: 30 minutes, 1 hour, 1 hour 30 minutes, 2 hours, 2 hours 30 minutes \& 3 hours).}
  \label{figure:finalClustering3h-6}
\end{figure}


\clearpage




\subsection{Clustering evaluation results}
\label{appendix:clusteringEvaluationResults}
% In the following figures, the green highlighted values indicate the best achieving evaluation score values (1h or 3h files), for the corresponding clustering method. Furthermore, the dark green highlighted values also accentuate the overall best scoring values over all datasets (1h and 3h files).


\begin{figure}[H]
  \centering
  \includegraphics[width=0.8\textwidth]{./images/clusteringResults/clusteringResults1.png}
  \caption{Evaluation scores comparison from the \textbf{first run} of t-SNE and clustering with a \textbf{learning rate of 20}.}
  \label{figure:clusteringResults1}
\end{figure}

\begin{figure}[H]
  \centering
  \includegraphics[width=0.8\textwidth]{./images/clusteringResults/clusteringResults2.png}
  \caption{Evaluation scores comparison from the \textbf{second run} of t-SNE and clustering with a \textbf{learning rate of 20}.}
  \label{figure:clusteringResults2}
\end{figure}


\begin{figure}[H]
  \centering
  \includegraphics[width=0.8\textwidth]{./images/clusteringResults/clusteringResults3.png}
  \caption{Evaluation scores comparison averaged from \textbf{figures \ref{figure:clusteringResults1} and \ref{figure:clusteringResults2}}.}
  \label{figure:clusteringResults3}
\end{figure}

\begin{figure}[H]
  \centering
  \includegraphics[width=0.8\textwidth]{./images/clusteringResults/clusteringResults4.png}
  \caption{Evaluation scores comparison \textbf{averaged} from \textbf{2 runs} of t-SNE and clustering with a \textbf{learning rate of 20}.}
  \label{figure:clusteringResults4}
\end{figure}

\begin{figure}[H]
  \centering
  \includegraphics[width=0.8\textwidth]{./images/clusteringResults/clusteringResults5.png}
  \caption{Evaluation scores comparison \textbf{averaged} from \textbf{2 runs} of t-SNE and clustering with a \textbf{learning rate of 800}.}
  \label{figure:clusteringResults5}
\end{figure}

\begin{figure}[H]
  \centering
  \includegraphics[width=0.8\textwidth]{./images/clusteringResults/clusteringResultsPlaces.png}
  \caption{Evaluation scores comparison to determine \textbf{2nd, 3rd, 4th, 5th, and 6th place}.}
  \label{figure:clusterResultsPlaces}
\end{figure}

\begin{figure}[H]
  \centering
  \includegraphics[width=0.8\textwidth]{./images/clusteringResults/clusteringResultsPlaces2.png}
  \caption{Evaluation scores of direct comparison of \textbf{30 min (1h), 1h (1h), and 1h (3h)}.}
  \label{figure:clusterResultsPlaces2}
\end{figure}


\clearpage








%\renewcommand{\thesubsection}{\Alph{subsection}}

\section{git-Repository}

% \url{https://gitlab.mediacube.at/fhs41216/BacThesis}
\textbf{todo: list contents of git repo}

Das Repository dient zur Dokumentation und Nachvollziehbarkeit der Arbeitsschritte. Stellen Sie sicher, dass der/die BetreuerIn Zugriff auf das Repository hat. Stellen im Sinne des Datenschutzes sicher, dass das Repository nicht für andere zugänglich ist.

Verpflichtende Daten für Bachelorarbeit 1 und 2:

\begin{itemize}
	\item LaTeX-Code der finalen Version der Arbeit
	\item alle Publikationen, die als pdf verfügbar sind.
	\item alle Webseiten als pdf
\end{itemize}

Verpflichtende Daten für Bachelorarbeit 2:
\begin{itemize}
	\item Quellcode für praktischen Teil
	\item Vorlagen für Studienmaterial (Fragebögen, Einverständniserklärung, ...)	
	\item eingescanntes, ausgefülltes Studienmaterial (Fragebögen, Einverständniserklärung, ...)
	\item Rohdaten und aufbereitete Daten der Evaluierungen (Log-Daten, Tabellen, Graphen, Scripts, ...)	
\end{itemize}

Link zum Repository auf dem MMT-git-Server {\url{gitlab.mediacube.at}}:

{\color{red}\url{https://gitlab.mediacube.at/fhs41216/BacThesis}}
	
% \section{Vorlagen für Studienmaterial}

% Vorlagen für Studienmaterial müssen in den Anhang. 

\section{Archivierte Webseiten}
% \show\UrlBreaks
\sloppy
\url{http://web.archive.org/web/20160526143921/http://www.gamedev.net/page/resources/_/technical/game-programming/understanding-component-entity-systems-r3013}, letzter Zugriff 1.1.2016

\url{http://web.archive.org/web/20160526144551/http://scottbilas.com/files/2002/gdc_san_jose/game_objects_slides_with_notes.pdf}, letzter Zugriff 1.1.2016

\end{appendices}
