\documentclass[12pt,a4paper]{article}
\usepackage[includeheadfoot,margin=2.5cm]{geometry}

\usepackage{times}
\usepackage[utf8]{inputenc}
\usepackage{listings}

\usepackage{csquotes}
\usepackage{abbrevs}
\usepackage[authordate,bibencoding=auto,strict,noibid,backend=biber]{biblatex-chicago}
\bibliography{Bibliography}

% change language settings here "ngerman", "english"
\usepackage[english,ngerman]{babel}

\newabbrev{\authorid}{1710601007}
\newabbrev{\authorname}{Natasha Lauren Troth }
\newabbrev{\authormail}{ntroth.mmt-b2017@fh-salzburg.ac.at}
\newabbrev{\exposedate}{31. January 2020}
\newabbrev{\titlename}{Exposé ???}
\newabbrev{\supervisor}{FH-Prof. DI Dr. Simon Ginzinger, MSc}
\newabbrev{\address}{FH Salzburg}
\newabbrev{\thesisdate}{Salzburg, Austria, 17.April 2017}


\title{\titlename}

%\author{ \authorname\\ \scriptsize \authormail \\ \scriptsize \address }
\author{ \authorid\\ \scriptsize \address }


\date{\exposedate}


\begin{document}
\selectlanguage{english}

\maketitle

\section*{Concept}

%TODO: RELATED WORK!!!!!
% the book
\textcite{han2011data}[17, 18] declare that data mining is used to discover patterns and knowledge from data. This includes cleaning data, combining multiple sources, selecting and transforming relevant data, and extracting and evaluating data patterns.

Cluster Analysis is a type of machine learning algorithm known as unsupervised machine learning. It is used in data mining to divide data into groups (clusters). Each cluster contains data that is similar to each other, but dissimilar to the data allocated to other clusters. Cluster Analysis can be used to acquire knowledge on the distribution of the data, discover characteristics, detect outliers and reduce noise, or to pre-process data for other algorithms \autocite{han2011data}[32, 362, 363, 367]. According to \textcite{bramer2007principles}[311], the grouping of similar attributes is applied in various fields such as economics, marketing, medicine, crime analysis and more.
%end


% %book - page 366-368, 373, 374, 385, 392, 414
% There are several different methods to create clustering. \textcite{han2011data}[366-368, 373, 374, 385, 392]  divide them into the following categories:
% \begin{itemize}
% 	\item Partitioning methods (examples: k-means, k-medoids)
% 	\item Hierarchical methods (examples: BIRCH, Chameleon)
% 	\item Density-based methods (examples: DBSCAN, OPTICS)
% 	\item Grid-based methods (examples: STING, CLIQUE)
% \end{itemize}
% %end book

%Many clustering algorithms determine clusters based on Euclidean or Manhattan distance measures
%book - page 366-368, 373, 374, 385, 392, 414
There are several different methods to create clustering. \textcite{han2011data}[362, 364, 366-368, 373, 374, 385, 392] explain, that objects are often arranged into clusters using distance measures (e.g. Euclidean or Manhatten distance measures). The authors divide the clustering algorithms into the following categories:
\begin{itemize}
	\item Partitioning methods: The data is divided into \textit{k} (generally pre-defined) number of groups. A data object can only be classified into one group (fuzzy partitioning methods relax this condition). Examples: k-means, k-medoids
	\item Hierarchical methods: Data is grouped into a hierarchy of clusters. In one approach, each object creates its own cluster and is then merged into its neighbours until all objects belong to one cluster (agglomerative or bottom-up approach). In the other approach, all objects form one cluster and are then divided until each object is contained in its own cluster (divisive or top-down approach). Examples: BIRCH, Chameleon
	\item Density-based methods: While partitioning and hierarchical methods only find clusters with spherical shapes, this method finds clusters with random shapes. It can also remove noise and outliers. Examples: DBSCAN, OPTICS
	\item Grid-based methods: The objects are quantised into grid cells. The operations are performed on the grid structure. This leads to an accelerated processing time. Examples: STING, CLIQUE
\end{itemize}
%end book
\textcite{feldman2007text}[92] outline that the results of clustering need to be judged by a human, thus however introducing subjectivity.

% Talk about SmartEater.
SmartEater \footnote{\url{https://sites.google.com/site/eatingandanxietylab/resources/smarteater}} is an upcoming mHealth (mobile health) app, with the goal to provide the user with content-dependent feedback, to avert a food craving episode. The app will predict future eating crises based on the user's past behaviour. In order to reduce intense user input, the app records and uses various smartphone sensor data.  With the help of data mining, machine learning algorithms and pattern recognition, this recorded situational context data will aid in predicting stress. The following data is recorded by the app:

\begin{enumerate}
	\item Background volume
	\item Relative movement of the smartphone (gyro and accel)
	\item Time and duration of phone calls (without storing the numbers)
	\item Time of messages (e.g. SMS, WhatsApp) (without collecting identifying information such as content, addresses, numbers)
	\item Screen activity (so-called touch events)
	\item Screen-on-time (illuminated display)
	\item Ambient brightness
	\item Data volume per unit of time (summary value of all smartphone activities on the internet)
	\item Switch-on and switch-off times of the smartphone
\end{enumerate}



%cite smarteater website  - as footnote
This sensor data will be recorded for different lengths of time. It is necessary to determine which time period will be most fitting to make accurate predictions for the future. This thesis will use cluster analysis to determine which time period is most significant.

According to \textcite{han2011data}[414], the above-mentioned clustering methods work well with data sets that are not high-dimensional an have less than 10 attributes. Since the SmartEater data set only has 9 dimensions, it is not considered high-dimensional. This paper will therefore utilise these clustering methods. Since different clustering algorithms can yield different results, multiple methods will be used and compared.
%book page 93
To reduce the size and amount of data, dimensionality reduction will be used. \textcite{han2011data}[93] define dimensionality reduction as a type of data reduction, which removes random attributes and creates a smaller data set with close to equal integrity. This thesis will use principal component analysis (PCA) to reduce the dimensionality.
%Visualizing Data using t-SNE p2579 (abstract!)
Furthermore, T-Distributed Stochastic Neighbor Embedding (t-SNE) will be employed to depict the data set in this thesis. \textcite{maaten2008visualizing}[2579] first introduce t-SNE, which is used to visualise data with a higher dimensionality. 

The clustering methods will be implemented using a Python machine learning platform or library (e.g. Anaconda\footnote{\url{https://www.anaconda.com/}}, scikit-learn\footnote{\url{https://scikit-learn.org/stable/}}). Next these will be implemented on the other time lengths. The resulting clusters of each time length will be compared to one another and evaluated. Evaluation examples given by \textcite{han2011data}[396-399] include clustering tendency, intrinsic and extrinsic measurements. 

The introduction of the thesis will serve as an overview of the SmartEater project and explain how and why the subsequent experiment will be conducted. The following chapter will concentrate on the theory of data mining and cluster analysis. After covering these topics, the next section will describe the conducted experiment and its results. The conclusion will summarise the findings of the experiment.


% Write something here about how it is hard to determine which clustering method is the best

%TODO: explain what methods i will use
%say how have to use different ones to compare




%Springer, principles of data mining book - 311 - 313
%According to /textcite, clustering is used to class similar objects together, and dissimilar objects into other classes. 
% \textcite{bramer2007principles}[312, 313] explains, that data with a maximum number of 3 attibutes (dimensions) can easily be visualised, as can the resulting clusters. Often there is a higher number of attibutes, which is impossible to visualise.
%end



%Explain the experiment and explain which python toolkits will be used
% In order to determine how long the smartphone sensor data should be recorded, to receive the best clustering results, the following experiment will be conducted: (--- TODO - prepare the data types). Initially the data for one chosen time length will be be clustered using different clustering methods. These algorithms (--- TODO - say which) will be implemented using the Python library.... (--- TODO - SAY WHICH ONE). Using principal component analysis (PCA), the initial data of the same time length will be reduced and the clustering methods reimplemented. These methods will be reproduced on the other time lengths and compared using ...(--- TODO - how to compare), potentially thus resulting in the time length with the clearest clusters. 

% The thesis will be arranged as following:
%outline paper


%python libs:
% pyclustering
% scikit
% RPy




%how to handle different data types - say what kind of data there is

%add more papers (different papers)

%how to evaluate if time period cluster is good
% talk about how to evaluate the clusters
% %page 3
% Different types of data implemented in clustering include interval-scaled variables, binary variables, nominal, ordinal, mixed and ratio variables.



% Why am I clustering data -> To predict similarities - similar behaviour which can then be used to create a pattern to predict stress - also to remove outliers (remove noist/clean the data) - curse of dimensionality
% - clusters that cause stress and clusters that don't

% Wie findet man sinnvolle Zeitslots?
% Wie evaluiert man, ob ein Zeitslot sinnvoll ist?
% Welcher Classifier wird verwendet?
% Unbegrenzte Nummer an Gruppen
% Ausreißer: Muss ein Datensatz zu einer Gruppe gehören? Neue Gruppe?
% Rocchio Classification? KNN Classification?
% Risiko: Zu genaue / zu ungenaue Klassifizierung



% Highlight outline of paper

%predicting weather
% While doing cluster analysis, we first partition the set of data into groups based on data similarity and then assign the labels to the groups.


% This thesis will focus on declaring the most fitting length of time period to log the smartphone data, in order to gain clear clusters.


% If you mention Web pages that are not proper scientific sources, just put the URL in a footnote, e.g. documentation for \LaTeX\footnote{\url{http://en.wikibooks.org/wiki/LaTeX}}
% or when refering to an online service like  sharelatex\footnote{\url{https://de.sharelatex.com/}}.





% % Wörtliches Zitat:
% % select proper language!
% % \selectlanguage{ngerman}
% \selectlanguage{english}
% \begin{quote}
% ``Erwin Unruh discovered that templates can be used to compute
% something at compile time. [...] The intriguing part of this exercise, however, was that the production of the prime numbers was performed by the compiler during the compilation process and not at run time.''
% \autocite[305]{Vandevoorde:2002}
% \end{quote}
% %select English again
% \selectlanguage{english}

% \textcite[]{Vandevoorde:2002} 

% \autocite[]{McConnell:2004:CCS:1096143,Vandevoorde:2002}.

\section*{Research Question}
What is the ideal length of time to record smartphone sensor data, in order to construct distinct clusters?

\section*{Outline}

\begin{enumerate}
	\item Introduction
	\item Theory
	\begin{enumerate}
		\item Data mining 
		\item Cluster analysis
		\begin{enumerate}
			\item Overview of clustering algorithms
			\item Dimensionality reduction
		\end{enumerate}
	\end{enumerate}
	\item Experiment
	\begin{enumerate}
		\item Preparation of the data set
		\item Clustering
		\item Clustering after dimensionality reduction
		\item Comparison and evaluation of clusters of different time lengths
	\end{enumerate}
	\item Conclusions
\end{enumerate}

% nocite print the whole bibliography. Remove nocite to
% print only the cited references.
\nocite{*}
\printbibliography

% section* means that there will be no section numbering
\section*{Schedule}

\begin{itemize}
	\item 31st January 2020 - Hand in this exposé
	\item February 2020 - Read papers and do research
	\item 24th February 2020 - Upload the final exposé onto FHSys
	\item March 2020 - Meet with supervisor, read literature, analyse and experiment with clustering algorithms and write a rough draft
	\item April 2020 - Meet with supervisor, finish the paper and print and review details
	\item 10th May 2020 - Submission of the bachelor thesis
\end{itemize}

\section*{Supervisor}

I have discussed the thesis with FH-Prof. DI Dr. Simon Ginzinger, MSc. He is working on the SmartEater research project and suggested this subject to me.

\end{document}
